%!TEX root = ../dissertation.tex
\chapter{Conclusion}
\label{conclusion}

\indent We performed a search for the superpartner to the top quark using $\intlumi$ $\ifb$ of $13 \tev$ LHC data collected by the ATLAS detector.   The analysis was able to rule out stops between masses of 225 and 600 $\gev$ if the $m_{\stop} - m_{ninoone} = m_{t}$ and the stop decayed with 100 percent branching ratio to $\stop \rightarrow \ninoone + t$.   No previous analysis was able to gain sensitivity to this experimentally difficult region of phase space.  \\

\indent Many R-parity SUSY searches focus on events with large amounts of $\met$.  This strategy fails in this region of parameter space because the stop decays impart low amounts of momentum onto the neutralinos.  The low $\met$ distribution in signal makes it difficult to distinguish between stops and the dominant SM $\ttbar$ background.  We demonstrated that by targeting events with strong initial state radiation and using the correlations between ISR and $\met$ we are able to separate signal from $\ttbar$ and other SM background.  Specifically the ratio between $\met$ and ISR $\pt$ called $\RISR$ was found to peak at $m_{ninoone}/m_{\stop}$ for signal.  The width of the peak is only 8 percent including detector resolution effects.  \\

\indent As part of this search, we developed a new and accurate ISR identification algorithm.  The algorithm uses the axis of maximum back-to-back $\pt$ called the thrust axis, to separate the event into an ISR hemisphere and a sparticle hemisphere.  In events with strong ISR, the back-to-back recoil between the particles produced in the hard scattering event and ISR should represent the single largest back-to-back kick in the event.  Therefore, the axis of maximum back-to-back $\pt$, the thrust axis, will mimic the axis of the back-to-back recoil between the particles produced in the hard scattering interaction and the total ISR system.  The hemisphere containing the $\met$ should contain mainly the stop decay products since the neutralino are traveling in the same direction.  The hemisphere without $\met$ contain mainly ISR jets.  \\

\indent We were able to achieve 9 percent uncertainty on ISR $\pt$ in both stop signal and $\ttbar$ background if the event contained at least 400 $\gev$ of true ISR $\pt$.  The uncertainty includes detector resolution effects and is derived using Monte Carlo simulation.  This ISR identification method is completely generalizable for other new BSM particles and SM processes.  \\

% For stop signal, the thrust axis approximates the axis of recoil between the stops and the ISR.  For ttbar produced with strong ISR, the thrust axis approximates the recoil between ttbar and ISR.  \\

\indent Using properties of both the sparticle and ISR hemispheres and the correlation between ISR and $\met$ we are able to achieve better than $2\:1$ signal to background for stop masses between $250$ and $400$ $\gev$.  The background is 60-80 percent SM $\ttbar$ depending on the $R_{ISR}$ region.  The signal region specifically targets events with strong ISR in both signal and background and 90 percent of all $\ttbar$ which survives the signal region selection have greater than $ 400 \gev$ of true ISR $\pt$. Other backgrounds include $W$+jets, $Z$+jets, and single top at high $\RISR$ and QCD multijet at low $\RISR$. \\

\indent Total background systematic uncertainties are between 15-25 percent in bins with appreciable expected background statistics.  Only $\sim10$ percent is associated with the systematic uncertainty on ISR/FSR generation on $\ttbar$.  The low ISR/FSR uncertainties in the signal region is due to the well designed $\ttbar$ control region.  \\

\indent Similar to the signal region, the $\ttbar$ control region also selects for primarily $\ttbar$ plus at least 400 $\gev$ of ISR $\pt$.  There is no extrapolation across ISR $\pt$ between the $\ttbar$ control region and the signal region and the distributions of true ISR $\pt$ is nearly identical between signal and control for SM $\ttbar$.  The lack of extrapolation means the $\ttbar$ control region directly measures the amount of $\ttbar$ plus strong ISR in data instead of relying on theoretical calculations on the $\ttbar$ ISR $\pt$ distribution.  \\

\indent Another reason that the uncertainty on ISR and partons showers is small is because the ISR identification algorithm identifies entire ISR systems instead of individual jets.  As such, the algorithm is insensitive to uncertainties associated with an energetic ISR parton splitting into multiple jets during fragmentation and hadronization. \\

\indent The other large systematic uncertainties for this analysis include uncertainties on $\ttbar$ matrix element and parton shower calculations and theory uncertainty between the interference of SM single top and SM ttbar, each at about 5-10 percent.    \\

\indent This analysis serves also as a demonstration of the general strategy of using events with strong ISR to search for other BSM signatures with $\met$.  The correlations between $\met$ and ISR in compressed regions are dictated by special relativity alone.  As long as decay products gain most of their momenta from ISR, the correlations will be strong regardless of the specific particle in the hard scattering interaction. \\

\indent The ISR identification algorithm is also completely generalizable.  The thrust axis will mimic the back-to-back recoil between ISR and hard scattering particles so long as the ISR is strong enough to be the largest single back-to-back kick in the event.  This gives potential application to other searches for SUSY including searches on Higgsinos, Charginos, and other ISR assisted searches such as the mono-jet/mono-photon searches for dark-matter.  Even if the thrust based algorithm doesn't improve separation power between signal and background when compared to other ISR identification searches, using the thrust based algorithm can still significantly reduce ISR/FSR uncertainties.  The thrust based ISR algorithm identifies entire ISR systems and is insensitive to uncertainties associated with a hard ISR parton splitting into multiple jets. \\

\indent At the same time, the accurate ISR identification algorithm can also be used to measure SM ISR $\pt$ spectrums.  The thrust based algorithm is able to separate $\ttbar$ produced with around $550 \gev$ of ISR $\pt$ from $\ttbar$ with little ISR pt as demonstrated in the SM $\ttbar$ control region.  Current ttbar ISR $\pt$ differential cross-section measurements at ATLAS first attempt to reconstruct tops and identify all non-top jets as ISR jets.  Hadronic top reconstruction efficiency is extremely correlated with top $\pt$ with a top reconstruction efficiency of only ~30\% at top $\pt = 200 \gev$.  This means that ttbar ISR $\pt$ measurements that require the reconstruction of tops are inherently biased towards events with high hadronic top $\pt$. \\

\indent Because the ISR algorithm primarily uses the thrust axis and properties of the entire $\ttbar$ system, we can avoid reconstructing individual tops but still identify a region with high purity of $\ttbar$ plus strong ISR.  This gives a more independent measurement of the $\ttbar$ ISR $\pt$ distribution and avoids any top reconstruction inefficiencies. \\
