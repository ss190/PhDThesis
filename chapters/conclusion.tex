%!TEX root = ../dissertation.tex
\chapter{Conclusion}
\label{conclusion}

\indent We performed a search for the superpartner to the top quark using $\intlumi$ $\ifb$ of $13 \tev$ LHC data collected by the ATLAS detector.   The analysis was able to rule out stops between masses of 225 and 600 $\gev$ if the $m_{\stop} - m_{ninoone} = m_{t}$ and the stop decayed with 100 percent branching ratio to $\stop \rightarrow \ninoone + t$.   No previous analysis was able to gain sensitivity to this experimentally difficult region of phase space.  Traditional analysis strategy that focused on large amounts of $\met$ failed due to low momenta imparted onto the neutralinos from the stop decays alone.  The lack of $\met$ made distinguishing between stops and the dominant SM ttbar background difficult.  We demonstrated that by targeting events with strong initial state radiation and using the correlations between ISR and $\met$ we are able to separate signal from ttbar and other SM background.  Specifically the ratio between $\met$ and ISR $\pt$ called $\RISR$ was found to peak at $m_{ninoone}/m_{\stop}$ for signal.  The spread of the peak is only 8 percent including detector resolution effects.  \\

\indent As part of this project, we developed a new and extremely accurate ISR identification algorithm.  We were able to achieve 9 percent uncertainty on ISR $\pt$ in both signal and ttbar background if the event contained at least 400 $\gev$ of true ISR pt.  The uncertainty includes detector resolution effects and is derived using Monte Carlo simulation.  The algorithm uses the axis of maximum back to back $\pt$ called the thrust axis, to separate the event into an ISR hemisphere and a sparticle hemisphere.  In events with strong ISR, the back to back recoil between the particles produced in the hard scattering event and ISR should represent the single largest back to back kick in the event.  Therefore, the axis of maximum back to back $\pt$, the thrust axis, will mimic the axis of the back to back recoil between the particles produced in the hard scattering interaction and the total ISR system.  The method is completely generalizable for both exotic signal and SM background.  For stop signal, the thrust axis approximates the axis of recoil between the stops and the ISR.  For ttbar produced with strong ISR, the thrust axis approximates the recoil between ttbar and ISR.  \\

\indent Using properties of both the sparticle and ISR hemispheres and the correlation between ISR and $\met$ in signal we are able to achieve better than 2 to 1 signal to background for stop masses between $250$ and $400$ $\gev$.  The background is 70-90 percent SM ttbar depending on the $R_{ISR}$ region.  The signal region specifically targets events with strong ISR in both signal and background and 90 percent of all ttbar which survives the signal region selection have $> 500 \gev$ of true ISR pt. Other background include $W$+jets, $Z$+jets, and single top at high $\RISR$ and QCD multijet at low $\RISR$. \\

\indent Total background systematics are between 15-25 percent in bins with appreciable expected background statistics.  Of this only 10 percent is associated with the systematic uncertainty on ISR/FSR generation on ttbar.  The low ISR/FSR systematic uncertainty is due to the control region in ttbar which directly estimates the amount of ttbar plus strong ISR pt.  There is no extrapolation across ISR pt between CR and SR and the distributions of true ISR pt is nearly identical between SR and CR for SM ttbar.  The other largest systematic uncertainties include uncertainties on ttbar matrix element and parton shower calculations and theory uncertainty between the interference of SM single top and SM ttbar, each at about 10 percent. The uncertainty on ISR and partons showers are also small due to the fact that the ISR identification algorithm identifies entire ISR systems instead of individual jets.  As such, the algorithm is insensitive to uncertainties associated with an ISR parton splitting into multiple jets during fragmentation and hadronization.   \\

\indent This analysis serves also as a demonstration of the general strategy of using events with strong ISR to search for exotic signatures with $\met$.  There will always be strong correlation between ISR and $\met$ as long as a particle's decay products gain little $\pt$ from the particle decay alone. ISR identification algorithm is also completely generalizable because the thrust axis will mimic the back to back recoil between ISR and any particles so long as the ISR is strong enough to represent the largest single back to back kick in the event.  This gives potential application to other searches in SUSY and exotic particles searches including searches on Higgsinos, Charginos, and other ISR assisted searches such as monojet searches for dark-matter.  Even if the algorithm doesn't improve separation power between signal and background when compared to other ISR based searches, we still expect to have significantly smaller ISR/FSR uncertainty because the ISR algorithm identifies entire ISR systems and is insensitive to uncertainties associated with a hard ISR parton splitting into multiple jets.\\

\indent At the same time, the accurate ISR identification algorithm can also be used to identify SM processes with strong ISR pt.  The algorithm is able to separate ttbar produced with at least $500 \gev$ of ISR pt from ttbar with little ISR pt as demonstrated in the SM ttbar CR.  This represents a new method to measure SM differential cross-sections.  Current ttbar ISR pt differential cross-section measurements at ATLAS first attempt to reconstruct tops and identify all non-top jets as ISR jets.  Hadronic top reconstruction efficiency is extremely correlated with top $\pt$ with a top reconstruction efficiency of only ~30\% at top $\pt = 200 \gev$.  This means that ttbar ISR $\pt$ measurements that require the reconstruction of tops are inherently biased towards events with high hadronic top $\pt$. \\

\indent Because the ISR algorithm primarily uses the thrust axis and properties of the entire di-top system, we can avoid reconstructing individual tops but still identify a region with high purity of ttbar plus strong ISR.  This gives a more independent measurement of the ttbar ISR $\pt$ distribution and avoids any top reconstruction inefficiencies. \\
