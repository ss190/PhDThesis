
\chapter{Theoretical Motivation}
\label{chap:motivation}
\section{The Standard Model}

\indent  The standard model (SM) describes our current understanding of the interactions of all known elementary particles.  SM is composed of 3 parts; fermions with spin 1/2 that make up the visible matter in our universe; vector bosons with spin 1 that mediates the interactions between fermions; and the scalar spin 0 Higgs boson that gives mass to the massive fermions and some vector bosons.  The fermions are organized in two groups, the quarks and leptons, with three families of increasing mass.  The force mediators, the photon, $W/Z$ boson, and gluon are respectively responsible for the electromagnetic, weak, and strong interactions.  A diagram listing all known particles is shown in figure \ref{fig:SM:part}. \\

\begin{figure}[htbp]
	\begin{center}
		\includegraphics[width=0.85\textwidth]{figures/theory/Standard_Model_of_Elementary_Particles.png}
		\caption{List of standard model elementary particles}
		\label{fig:SM:part}
	\end{center}
\end{figure}

\indent  Interactions in the SM are described by non-abelian Yang-Mills gauge theory with the gauge group $SU(3)_C \times SU(2)_L \times U(1)_Y$ where $SU(3)_C$ corresponds to the strong interaction and $SU(2)_L \times U(1)_Y$ corresponds to the electroweak interactions. SM particles can be  organized by the representations of the gauge groups.  The left handed component of the fermions form an $SU(2)_L$ doublet while the right handed handed components form an $SU(2)_L$ singlet.  Effectively this means only the left handed component of SM fermions interacts via the weak interaction.  \\

\indent In addition to EW interactions,  the quarks also interact via the strong interaction described by the $SU(3)_C$ symmetry.  These quarks carry color charge in addition to their electromagnetic charges.  The gluons which mediate the strong interactions also carry color charge.  The self interaction of the gluon causes the coupling strength of the strong coupling constant $\alpha_s$ to diverge at low energies.  This phenomena called confinement ensures that no free quarks are confined to be within composite color singlet states in the form of hadrons.  At the same time, the running of $\alpha_s$ approaches zero at high energy forming a phenomenon known as asymptotic freedom. \\

\indent For energetic particles like those produced in proton-proton collisions at the LHC, colored partons will radiate additional collinear gluons and quark/anti-quark pairs in a parton shower. These partons will in-turn form color-singlet hadrons once the energy scale is lower then IR-cutoff scale due to confinement.  The result is a jet of color-neutral baryons and mesons localized in a narrow cone in the direction of the initial colored parton. \\

\indent The generators of the gauge groups correspond to the massless spin 1 vector bosons.  However, the $W^\pm$ and $Z$ bosons acquire mass through spontaneous electroweak symmetry breaking using the Higgs mechanism.  This is accomplished using an additional complex $SU(2)_L$ doublet of spin zero field, the Higgs field.  The Higgs has a nonzero vacuum expectation value (VEV) at the minimum of its quadratic potential shown in equation \ref{eqn:Higgs}.  When $\lambda > 0$ and $m_H^2 < 0 $, $\braket{H} = \sqrt{-m_H^2/2\lambda}$.  \\

\begin{equation}
\label{eqn:Higgs}
V(H) = m_H^2 |H|^2 +\lambda |H|^4
\end{equation}

\indent This breaks the $SU(2)_L \times U(1)_Y$ electroweak symmetry and leaves only the $U(1)_{em}$ electromagnetism invariant.  Meanwhile, the other gauge bosons from $SU(2)_L \times U(1)_Y$ gains a longitudinal degree of freedom from degrees of freedom associated with the Higgs doublet and thereby gaining mass.  The photon, $W^\pm$ and $Z$ bosons are therefore linear combinations of the original $SU(2)_L and U(1)_Y$ generators.  The Higgs boson also gives fermions their mass through Yukawa couplings. \\

\indent After symmetry breaking, only one neutral scalar component of the Higgs doublet is left.  This is the massive Higgs boson observed in July 2012 at the LHC.  \\

\section{Introduction to Super-Symmetry}
\label{Theory:QFT}

\indent Current combined measurement of the Higgs boson at ATLAS and CMS gives an observed Higgs mass of $125.09\pm0.21$(stat)$\pm0.11$(syst) GeV.\cite{Higgs2016}  Plus $\braket{H} \sim 174 \gev$ due to experimental measurements of the properties of the weak interactions.  This implies that the parameters $\lambda$ and $m_H^2$ in the Higgs potential in equation \ref{eqn:Higgs} have the values of $0.126$ and $-(92.9 \gev)^2$ assuming SM is the correct effective field theory.  \\

\indent However theoretical calculations gives enormous quantum corrections to $m_H^2$.\cite{MartinSUSY}  For example, the correction to $m_H^2$ from a loop containing a Dirac fermion $f$ with mass $m_f$ is given in equation \ref{eqn:Higgs:Loopf}.  The Feynman diagram associated with the fermion loop is shown in figure \ref{fig:Higgs:Loopf} \\

\begin{equation}
\label{eqn:Higgs:Loopf}
\Delta m_H^2 = - \frac{|\lambda_f|^2}{8\pi^2}\Lambda^2_{UV} + ....
\end{equation}

\begin{figure}[htbp]
	\begin{center}
		\includegraphics[width=0.45\textwidth]{figures/theory/loopf.png}
		\includegraphics[width=0.45\textwidth]{figures/theory/loopS.png}
		\caption{One-Loop corrections due to a Dirac fermion $f$ and a scalar $\tilde{f}$ to the Higgs mass parameter $m_H^2$}
		\label{fig:SM:Loopf}
	\end{center}
\end{figure}

\indent $\lambda_f$ is the Yukawa coupling between the fermion and the Higgs and $\Lambda_{UV}$ is the ultraviolet cutoff used to regulate the loop integral.  $\Lambda_{UV}$ can be interpreted as around the energy scale of new physics.  Since the scale of new physics maybe orders of magnitudes larger then the electroweak scale, the quadratic dependence of $m_H^2$ on $\Lambda_{UV}$ makes the Higgs potential extremely sensitive to new physics and fine tuning between the Higgs bare-mass and the corrections is needed to keep the observed $m_H^2 = -(92.9 \gev)^2$.  This sensitivity to high mass scales for the Higgs potential is referred to as the hierarchy problem.  Additional terms also exists in the correction but they grow at most logarithmically in $\Lambda_{UV}$. \\

\indent Supersymmetry (SUSY) solves this problem by proposing that there exist a new space-time symmetry with respect to the transformation $Q$ that turns fermions into bosons and bosons into fermions.\\

\begin{equation}
\label{eqn:Higgs:Loopf}
Q\ket{Boson} = \ket{Fermion} ~~~~~~~~~~~~~~~~~~~~~~~~~~~~ Q\ket{Fermion} = \ket{Boson}
\end{equation}

\indent The supersymmetric Lagrangian is invariant under transformations of $Q$ and $Q^{\dagger}$.  In order for this to be satisfied, SUSY proposes the existence of a supersymmetric partner (superpartner) to every known SM particle which is related to each other by the Q transformation and differ from each other by spin $1/2$.  If SUSY was an exact symmetry then, the SM particle and its superpartner must have the same mass as Q does not change mass.  However, we have yet to discover even a single superpartner to the SM.  Therefore, SUSY must be broken at low energy scales and the superpartners have significantly more mass then their SM counter parts.  \\

\indent Supersymmetry symmetry breaking can occur in many ways that are beyond the scope of this document.  For example in Gauge-mediated supersymmetry breaking, some scalar fields in the SUSY Lagrangian gains a vacuum expectation value due to their potential energy shape.  This symmetry breaking gives mass to some fermions and their super-partners called messengers.  The messengers are too heavy to be directly detectable, but they give mass to the superpartners of the SM via loop interactions.  SM gauge symmetry ensures that the loop correction to the SM gauge bosons are zero to all orders of magnitudes, but the same protection is not afforded to their superpartners, the gauginos, which gains mass through one-loop diagrams involving virtual messenger particles.  Likewise, the scalar partners to SM fermions gain mass through two-loop diagrams involving virtual messengers and SM gauge bosons.  These loop contributions leads to heavier superpartners relative to their SM counter parts.  More details on SUSY symmetry breaking can be found in \cite{MartinSUSY}. \\

\indent For now we will look at some phenomenological consequences of the existence of heavy superpartners.  For example, if a complex scalar particle $\tilde{f}$  with mass $m_{\tilde{f}}$ exists and couples to the Higgs with the term $-\lambda_S|H|^2|\tilde{f}|^2$ then correction due to the loop diagram in figure \ref{fig:SM:Loopf} is given in equation \ref{eqn:Higgs:LoopS}. \\

\begin{equation}
\label{eqn:Higgs:LoopS}
\Delta m_H^2 = \frac{\lambda_s}{16\pi^2}[\Lambda^2_{UV} - 2m_{\tilde{f}}^2 \ln{\Lambda_{UV}/m_{\tilde{f}}}+ ....]
\end{equation}

\indent This correction also contains a quadratically divergent term that has an opposite sign to equation \ref{eqn:Higgs:Loopf}.  The two quadratic contributions to $m_H^2$ will cancel if $|\lambda_f|^2 = \lambda_s$ and we are left with only a term that is proportional to $\ln{\Lambda_{UV}/m_{\tilde{f}}}$. In fact, this cancellation of quadratically divergent term will occur not only for the 1 loop case, but for all orders of magnitude in perturbation theory if supersymmetry exists. \\

\indent The term that remains after cancellation is proportional to equation \ref{eqn:Higgs:logterm}. \\

\begin{equation}
\label{eqn:Higgs:LoopS}
\Delta m_H^2 \sim m_{\tilde{f}}^2[\frac{\lambda_s}{16\pi^2}\ln{\Lambda_{UV}/m_{\tilde{f}}}]
\end{equation}

\indent Its important to note that while the correction is now not so strongly dependent on $\Lambda_{UV}$ because of the natural log, the correction term is also directly proportional to $m_{\tilde{f}}^2$.  This implies that the superpartners masses cannot be too large, otherwise the correction to $m_H^2$ is again too large.  Setting $\Lambda_{UV}$ to approximately at the Planck scale $M_P$ and $\lambda_s \sim 1$, we find that $m_{\tilde{f}}$ for the lightest supersymmetric particle should not be heavier then the $\tev$ scale \cite{MartinSUSY} if SUSY is the solution to the hierarchy problem.  In particular, we know that the superpartner to the top quark has a coupling to the Higgs of order $1$ due to $\lambda_S = |\lambda_f|^2 \sim 0.94^2$. This makes the stop potentially within reach of the energy of the LHC. \\

\indent Supersymmetry also introduces many new interactions not found in the SM.  Some of these interactions directly violate total lepton and baryon numbers.  If such interactions existed then the half life of a proton may be a tiny fraction of a second.  However, proton decay experiments have shown that the proton half-life exceeds $10^{32}$ years.  A new discrete symmetry called {\it R-parity} is introduced to remove these B and L violating terms from the supersymmetric Lagrangian.  \\

\indent The quantity $P_R$ defined in equation \ref{eqn:PR} and must multiply to 1 for all interaction vertexes for R-parity to be conserved. $P_R$ equals $1$ for all SM particles and equals $-1$ for all superpartners.  \\

\begin{equation}
\label{eqn:PR}
P_R = (-1)^{3(B-L)+2s}
\end{equation}

\indent R-parity conservation has several important phenomenological consequences.  In R-parity respecting SUSY, superpartners are always produced in pairs.  Superpartners must always decay into other superpartners forming a long decay chain of SUSY particle to SUSY particle that ultimately end in the lightest supersymmetric particle (LSP) which is absolutely stable.  If the LSP is electrically and color neutral, then it is an attractive dark matter candidate as a weakly interacting, stable particle.  We assume R-parity is conversed and the LSP is a neutralino for this analysis.  \\
