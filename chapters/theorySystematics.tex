
\subsection{Theoretical Uncertainties}
\label{sec:TheoSystematics}

Theory uncertainties affecting the background normalization and kinematic distribution shapes largely impact the background prediction in the signal regions, as they directly affect the background normalization and acceptance times efficiency. If a background normalization is determined by making use of dedicated control regions, then only systematics affecting the analysis acceptance are relevant. Statistical uncertainties in the evaluation of systematics are neglected in general; where necessary, selection cuts are loosened to make the systematic comparison statistically meaningful. The remainder of this section is dedicated to the discussion on how the theory systematic uncertainties have been derived for each of the background processes considered.

The theoretical uncertainty in each signal region is evaluated by
considering variations with respect to the default settings and choices for the event generation. For each of the variations considered, the systematic uncertainty is estimated as an uncertainty on the so-called transfer factor, that is, the ratio of the predicted yields between the signal region and the $\ttbar$ control region(s). For a given control - signal region pair, the transfer factor $T_f$ is defined by:

\begin{equation}
  T_f = \frac{N_{\mathrm{SR}}}{N_{\mathrm{CR}}}
\end{equation}

\begin{description}
\item[\boldmath $Z+jets$] For the estimation of the $Z$ production background, the \sherpa\ generator is used by default. To assess the theory systematic uncertainties due to scale variations with good statistical power seven LHE3 weights are used. 

The theory uncertainty on the normalisation of the $Z$ production is obtained by comparing the \sherpa\ predictions on the transfer factor between the control region and the signal region with and without systematic weights applied. The uncertainty on the transfer factor is computed with the following equation: 
      \begin{eqnarray}
    \Delta_{X} = \frac{T_f^{\mathrm{up}} - T_f^{\mathrm{down}}}{T_f^{\mathrm{up}} + T_f^{\mathrm{down}}}
    \label{eq:theory_uncertainty}
  \end{eqnarray}

  , where $X$ is the systematic variation. Tables~\ref{tab:ZThSyst} summarises the values of the transfer factors obtained for \sherpa\ systematics for all signal regions, together with the resulting relative uncertainty.

  The $Z+jets$ theory uncertainty is denoted by {\bf theoSysZ} in the fit.

  \begin{table}[!h]
    \begin{center} \footnotesize
\begin{tabular}{c||c} \hline\hline
{\bf SR} & {\bf uncertainty (\%)} \\ \hline
SRA-TT & 35.7\\ \hline
SRA-TW & 36.2\\ \hline
SRA-T0 & 36.7\\ \hline
SRB-TT & 35.0\\ \hline
SRB-TW & 32.1\\ \hline
SRB-T0 & 28.4\\ \hline
SRD-low & 36.5\\ \hline
SRD-high & 36.5\\ \hline
SRE & 36.5\\ \hline
VRZAB & 22.0\\ \hline
VRZD & 34.6\\ \hline
VRZE & 31.4\\ \hline
\hline
\end{tabular}

    \end{center}
    \caption{Summary of the theory uncertainties (in percent) on $Z$ production obtained on the transfer factor for all signal regions except for SRC for which the Z background is negligible. The largest uncertainty is below 37\% in SRA-T0. }
    \label{tab:ZThSyst}
  \end{table}  
  

\item[\boldmath $W+jets$]
  A similar approach is performed to assess the $W$+jets theory uncertainties. No additional uncertainty due to possible $W+c$ contribution in CRW which has a one $b$-jet requirement (compared to the 2b requirement). It was found that the flavor contribution in the VRW (which also has a two b-jet requirement) is very similar to the SR. The postfit MC from VRW match the observed data well which leads us to believe that the $W+c$ contribution in CRW has a negligible effect on the normalization factor. The details are described in Section~\ref{sec:WCR}. 

Table~\ref{tab:WThSyst} summarises the values of the transfer factors obtained for \sherpa\ systematics for all signal regions, together with the resulting relative uncertainty.

  These $W$ theory uncertainties are denoted by {\bf theoSysW, theoSysWc} in the fit, respectively.


  \begin{table}[!h]
    \begin{center} \footnotesize
\begin{tabular}{|c|c|} \hline
{\bf SR} & {\bf uncertainty (\%)} \\ \hline
%SRA-TT & 9.5\\ \hline
%SRA-TW & 8.0\\ \hline
%SRA-T0 & 6.1\\ \hline
%SRB-TT & 9.1\\ \hline
%SRB-TW & 7.9\\ \hline
%SRB-T0 & 3.3\\ \hline
%SRC1 & 11.4\\ \hline
SRC1 & 12.5\\ \hline
SRC2 & 11.8\\ \hline
SRC3 & 10.7\\ \hline
SRC4 & 9.5\\ \hline
SRC5 & 11.3\\ \hline
%SRD-low & 8.8\\ \hline
%SRD-high & 8.2\\ \hline
%SRE & 9.5\\ \hline
%VRW & 1.9\\ \hline
\end{tabular}

    \end{center}
    \caption{Summary of the theory uncertainties (in percent) on $W$ production obtained on the transfer factor for all signal regions. 
      The largest uncertainty is 13\%. }
    \label{tab:WThSyst}
  \end{table}         

\item{\boldmath \bf \ttbar\ production} Most \ttbar\ uncertainties are estimated by comparing variations at reco level. The hard scattering generation uncertainty however is estimated using truth quantities. In this case, the yields are estimated by repeating the analysis at particle level. The object definitions for the particle level analysis are described in Appendix~\ref{sec:truthAna}.

  The recommended variations~\cite{ATLAS:ttbarTheoryUnc} of the event generation considered are summarized below, with details about how the corresponding uncertainty has been evaluated.  

  \begin{itemize}
  \item{\bf Hard scatter generation:} An uncertainty on the hard scattering generation is computed by comparing the \mcatnlo\pythia\ with DSID 410225 and \powheg\pythia\ with DSID 410000, 407012. In this case, the uncertainty is given by:

    \begin{eqnarray}
      \Delta_{\mathrm{hard~scatter}} = \frac{T_f^{\mathrm{\powheg}} - T_f^{\mathrm{\mcatnlo}}}{T_f^{\mathrm{\powheg}}}
    \end{eqnarray}

  A sample generated using \sherpa\ 2.2.1 was also considered for this uncertainty.

  \item{\bf Parton shower:} An uncertainty on the choice of the showering model is computed by comparing the baseline DSID 410000, 407012 (that is, \powheg\pythia\ with the P2011C tuning for \pythia, where the two DSIDs correspond to the inclusive and high \met -sliced samples) and DSID 410004 (that is, \powheg\herwigpp\ with the UE5C6L1 CT10 tuning for \herwigpp). In this case, the uncertainty is given by:

    \begin{eqnarray}
      \Delta_{\mathrm{PS}} = \frac{T_f^{\mathrm{\pythia}} - T_f^{\mathrm{\herwigpp}}}{T_f^{\mathrm{\pythia}}}
      \label{eq:ttbar_frag_uncertainty}
    \end{eqnarray}

  \item{\bf ISR/FSR:} Uncertainties related to the emission of additional partons in the initial or final states are evaluated by comparing the radHi and radLow \powheg\pythia\ samples with DSID 410002 and 410001. The uncertainty is calculated with eq~\ref{eq:theory_uncertainty}.

  \end{itemize}

  Tables ~\ref{tab:ttbar_unc_SRA_TT}, ~\ref{tab:ttbar_unc_SRA_TW}, ~\ref{tab:ttbar_unc_SRA_T0},  ~\ref{tab:ttbar_unc_SRB_TT},
~\ref{tab:ttbar_unc_SRB_TT},~\ref{tab:ttbar_unc_SRA_TW},~\ref{tab:ttbar_unc_SRB_T0},~\ref{tab:ttbar_unc_SRC},~\ref{tab:ttbar_unc_SRD}
and ~\ref{tab:ttbar_unc_SRE} summarise the $\ttbar$ systematic uncertainties obtained by the comparisons described above for the different
control, signal and validation regions in this analysis. Control regions in which the $\ttbar$ background is negligible are not considered. 
The results for the single top control region are found in Table~\ref{tab:ttbar_unc_SRB_T0}, while results for CRW are in
 Table~\ref{tab:ttbar_unc_SRE}.

 In general, limited statistics of the \mcatnlo\pythia\ and \sherpa\ samples result in very large values of the generator uncertainty. 
In the SRA signal regions, all $\ttbar$ systematic uncertainties are limited by statistics of the relevant samples. For regions where
sufficient statistics are available, the uncertainties typically range from a few \% to 30\%.
%Appendix~\ref{sec:truthAna} contains plots with $\met$ distributions compared for some of the cases just described.

  The $\ttbar$ theory uncertainty is denoted by {\bf theoSysTop} in the fit.  
  
  \begin{table}[!h]
    \begin{center} \footnotesize
      \begin{tabular}{|c|c|c|c|}
        \hline
        & CRA-TT & SRA-TT & VRTopATT\\
        \hline
        ttbar&   $69.5\pm 2.9 $&         $0.36\pm 0.20 $&        $55.0\pm 2.7 $\\
        ttbar (rad up)&          $81.1\pm 3.5 $&         $1.32\pm 0.35 $&        $80.0\pm 3.3 $\\
        ttbar (rad down)&        $58.8\pm 3.0 $&         $0.29\pm 0.13 $&        $48.9\pm 2.5 $\\
        ttbar (Powheg+H++)&      $60.1\pm 2.8 $&         $0.59\pm 0.20 $&        $57.2\pm 2.4 $\\
        ttbar (aMC@NLO+P8)&      $63\pm 16 $&    $<0.01$&        $32\pm 18 $\\
        ttbar (Sherpa)&          $73\pm 10 $&    $<0.01$&        $79\pm 11 $\\
        \hline
        \multicolumn{4}{c}{\bf Transfer factors (in \%)} \\ \hline
        ISR/FSR & &        $53$&   $9$\\
        PS & &        $90$&   $20$\\
        Generator (aMC@NLO) & & - & $40$\\
        Generator (Sherpa) & & - & $37$\\
        \hline
        \end{tabular}
    \end{center}
    \caption{Theory uncertainties for the $\ttbar$\ background for the SRA-TT regions.}
    \label{tab:ttbar_unc_SRA_TT}
  \end{table}

  \begin{table}[!h]
    \begin{center} \footnotesize
      \begin{tabular}{|c|c|c|c|}
        \hline
        & CRA-TW & SRA-TW & VRTopATW\\
        \hline
        ttbar&   $119.0\pm 3.5 $&        $0.66\pm 0.22 $&        $48.4\pm 2.7 $\\
        ttbar (rad up)&          $111\pm 4 $&    $0.43\pm 0.16 $&        $50.1\pm 2.6 $\\
        ttbar (rad down)&        $121\pm 4 $&    $0.59\pm 0.25 $&        $42.0\pm 2.0 $\\
        ttbar (Powheg+H++)&      $91.7\pm 3.0 $&         $0.61\pm 0.21 $&        $40.0\pm 2.3 $\\
        ttbar (aMC@NLO+P8)&      $101\pm 19 $&   $<0.01$&        $20\pm 11 $\\
        ttbar (Sherpa)&          $121\pm 19 $&   $0.5\pm 0.5 $&          $44\pm 8 $\\
        \hline
        \multicolumn{4}{c}{\bf Transfer factors (in \%)} \\ \hline
        ISR/FSR &      &  $11$&   $13$\\
        PS &  &      $20$&   $7$\\
        Generator (aMC@NLO) & & - & $51$\\
        Generator (Sherpa) & & - & $11$\\
        \hline
        \end{tabular}
    \end{center}
    \caption{Theory uncertainties for the $\ttbar$\ background for the SRA-TB regions.}
    \label{tab:ttbar_unc_SRA_TW}
  \end{table}

 \begin{table}[!h]
    \begin{center} \footnotesize
      \begin{tabular}{|c|c|c|c|}
        \hline
        & CRA-T0 & SRA-T0 & VRTopAT0\\
        \hline
        ttbar&   $76.2\pm 2.9 $&         $1.9\pm 0.6 $&          $58.3\pm 2.5 $\\
        ttbar (rad up)&          $70.1\pm 2.8 $&         $0.80\pm 0.29 $&        $56.8\pm 2.8 $\\
        ttbar (rad down)&        $85.1\pm 3.2 $&         $0.50\pm 0.19 $&        $63.9\pm 2.5 $\\
        ttbar (Powheg+H++)&      $60.2\pm 2.6 $&         $1.0\pm 0.5 $&          $42.4\pm 2.1 $\\
        ttbar (Sherpa)&          $61\pm 12 $&    $0.7\pm 0.7 $&          $53\pm 10 $\\     
        \hline
        \multicolumn{4}{c}{\bf Transfer factors (in \%)} \\ \hline
        ISR/FSR &     &   $32$&   $4$\\
        PS &  &      $30$&   $8$\\
        Generator (aMC@NLO) & & - & $68$\\
        Generator (Sherpa) & & - & $14$\\        
        \hline       
        \end{tabular}
    \end{center}
    \caption{Theory uncertainties for the $\ttbar$\ background for the SRA-T0 regions.}
    \label{tab:ttbar_unc_SRA_T0}
  \end{table}

 \begin{table}[!h]
    \begin{center} \footnotesize
      \begin{tabular}{|c|c|c|c|}
        \hline
        & CRB-TT & SRB-TT & VRTopBTT\\
        \hline
        ttbar&   $64.0\pm 2.8 $&         $6.5\pm 0.8 $&          $96.3\pm 3.5 $\\
        ttbar (rad up)&          $70.9\pm 3.0 $&         $11.2\pm 1.3 $&         $133\pm 5 $\\
        ttbar (rad down)&        $54.2\pm 2.5 $&         $4.3\pm 0.7 $&          $87.8\pm 3.4 $\\
        ttbar (Powheg+H++)&      $53.3\pm 2.3 $&         $7.5\pm 0.8 $&          $101\pm 4 $\\
        ttbar (aMC@NLO+P8)&      $58\pm 17 $&    $9\pm 11 $&     $48\pm 26 $\\
        ttbar (Sherpa)&          $66\pm 9 $&     $6.3\pm 2.6 $&          $123\pm 13 $\\
        \hline
        \multicolumn{4}{c}{\bf Transfer factors (in \%)} \\ \hline
        ISR/FSR &   &     $33$&   $7$\\
        PS &   &     $39$&   $26$\\
        Generator (aMC@NLO) &  & $50$ & $45$\\
        Generator (Sherpa) & & $10$ & $24$\\        
        \hline       
        \end{tabular}
    \end{center}
    \caption{Theory uncertainties for the $\ttbar$\ background for the SRB-TT regions. }
    \label{tab:ttbar_unc_SRB_TT}
  \end{table}


 \begin{table}[!h]
    \begin{center} \footnotesize
      \begin{tabular}{|c|c|c|c|}
        \hline
        & CRB-TW & SRB-TW & VRTopBTW\\
        \hline
        ttbar&   $265\pm 6 $&    $12.8\pm 1.1 $&         $119\pm 4 $\\
        ttbar (rad up)&          $241\pm 6 $&    $18.5\pm 2.1 $&         $129\pm 5 $\\
        ttbar (rad down)&        $275\pm 6 $&    $10.7\pm 1.1 $&         $121\pm 4 $\\
        ttbar (Powheg+H++)&      $200\pm 4 $&    $11.6\pm 1.3 $&         $107\pm 4 $\\
        ttbar (aMC@NLO+P8)&      $256\pm 33 $&   $6\pm 9 $&      $88\pm 24 $\\
        ttbar (Sherpa)&          $278\pm 26 $&   $16\pm 5 $&     $119\pm 13 $\\
        \hline
        \multicolumn{4}{c}{\bf Transfer factors (in \%)} \\ \hline
        ISR/FSR &   &     $33$&   $10$\\
        PS &   &     $20$&   $19$\\
        Generator (aMC@NLO) & & $50$ & $23$\\
        Generator (Sherpa) & & $20$ & $5$\\        
        \hline       
        \end{tabular}
    \end{center}
    \caption{Theory uncertainties for the $\ttbar$\ background for the SRB-TW regions.}
    \label{tab:ttbar_unc_SRB_TW}
  \end{table}

 \begin{table}[!h]
    \begin{center} \footnotesize
      \begin{tabular}{|c|c|c|c|c|}
        \hline
        & CRB-T0 & CRST & SRB-T0 & VRTopBT0\\
        \hline
        ttbar&   $432\pm 7 $&    $34.2\pm 2.1 $&         $45.8\pm 2.6 $&         $159\pm 5 $\\
        ttbar (rad up)&          $382\pm 13 $&   $39.2\pm 2.7 $&         $47.6\pm 3.0 $&         $139\pm 5 $\\
        ttbar (rad down)&        $457\pm 7 $&    $28.7\pm 1.9 $&         $36.6\pm 2.2 $&         $153\pm 5 $\\
        ttbar (Powheg+H++)&      $308\pm 6 $&    $27.9\pm 1.8 $&         $39.6\pm 2.4 $&         $122\pm 4 $\\
        ttbar (aMC@NLO+P8)&      $380\pm 40 $&   $59\pm 16 $&    $59\pm 20 $&    $100\pm 22 $\\
        ttbar (Sherpa)&          $376\pm 29 $&   $37\pm 8 $&     $58\pm 15 $&    $144\pm 17 $\\
\hline
        \multicolumn{5}{c}{\bf Transfer factors (in \%)} \\ \hline
        ISR/FSR &    &    $24$&   $22$&   $4.2$\\
        PS &   &    $14$&   $21$&   $8$\\
        Generator (aMC@NLO) &  &      $100$&          $50$&   $29$\\
        Generator (Sherpa) &   &     $24$&   $50$&   $4$\\        
        \hline       
        \end{tabular}
    \end{center}
    \caption{Theory uncertainties for the $\ttbar$\ background for the SRB-T0 regions.}
    \label{tab:ttbar_unc_SRB_T0}
  \end{table}


 \begin{table}[!h]
    \begin{center} \footnotesize
      \begin{tabular}{|c|c|c|c|c|c|c|c|}
        \hline
        & CRTopC & SRC1 & SRC2 & SRC3 & SRC4 & SRC5 & VRTopC\\
        \hline
ttbar&   $668\pm 9 $&    $16.7\pm 1.6 $&         $31.7\pm 2.1 $&         $21.7\pm 1.6 $&         $6.3\pm 0.8 $&          $0.60\pm 0.23 $&         $232\pm 5 $\\
ttbar (rad up)&          $872\pm 11 $&   $25.2\pm 2.3 $&         $39.5\pm 2.3 $&         $28.7\pm 2.1 $&         $8.6\pm 1.0 $&  $1.05\pm 0.33 $&         $293\pm 7 $\\
ttbar (rad down)&        $521\pm 9 $&    $10.1\pm 1.0 $&         $19.2\pm 1.6 $&         $15.8\pm 1.5 $&         $6.3\pm 1.2 $&  $0.7\pm 0.4 $&   $187\pm 5 $\\
ttbar (Powheg+H++)&      $621\pm 10 $&   $16.3\pm 1.8 $&         $27.8\pm 1.8 $&         $18.0\pm 1.5 $&         $6.5\pm 0.9 $&  $0.46\pm 0.18 $&         $206\pm 5 $\\
ttbar (aMC@NLO+P8)&      $310\pm 60 $&   $6\pm 5 $&      $<0.01$&        $4\pm 7 $&      $1\pm 5 $&      $0.9\pm 0.9 $&          $113\pm 34 $\\
ttbar (Sherpa)&          $840\pm 40 $&   $30\pm 8 $&     $42\pm 9 $&     $22\pm 5 $&     $7.4\pm 3.2 $&          $<0.01$&        $297\pm 30 $\\        
        \hline
        \multicolumn{8}{c}{\bf Transfer factors (in \%)} \\ \hline
        ISR/FSR &  &      $20$&   $10$&   $4$&    $10$&   $5$&    $3.3$\\
        PS &     &   $5$&    $6$&    $11$&   $11$&   $20$&   $4$\\
        Generator (aMC@NLO) &   &     $20$&   $110$&          $60$&   $70$&   $220$&          $0$\\
        Generator (Sherpa) &    &    $40$&   $5$&    $19$&   $10$&   $100$&          $2$\\
        \hline       
        \end{tabular}
    \end{center}
    \caption{Theory uncertainties for the $\ttbar$\ background for the SRC regions.}
    \label{tab:ttbar_unc_SRC}
  \end{table}

 \begin{table}[!h]
    \begin{center} \footnotesize
      \begin{tabular}{|c|c|c|c|c|c|}
        \hline
        & CRTopD & CRW & SRD-low & SRD-high & VRTopD\\
        \hline
        ttbar&   $134\pm 4 $&    $128\pm 4 $&    $3.2\pm 0.5 $&          $0.73\pm 0.23 $&        $159\pm 4 $\\
        ttbar (rad up)&          $143\pm 5 $&    $135\pm 5 $&    $5.0\pm 0.7 $&          $1.38\pm 0.35 $&        $164\pm 5 $\\
        ttbar (rad down)&        $132\pm 4 $&    $117\pm 4 $&    $2.6\pm 0.4 $&          $1.4\pm 0.4 $&          $148\pm 4 $\\
        ttbar (Powheg+H++)&      $93.0\pm 3.3 $&         $87.9\pm 3.1 $&         $3.6\pm 0.6 $&          $1.30\pm 0.33 $&        $117\pm 4 $\\
        ttbar (aMC@NLO+P8)&      $145\pm 22 $&   $218\pm 25 $&   $9\pm 5 $&      $3.5\pm 2.4 $&          $106\pm 32 $\\
        ttbar (Sherpa)&          $119\pm 15 $&   $131\pm 14 $&   $5.5\pm 2.9 $&          $1.5\pm 1.5 $&          $150\pm 14 $\\        
        \hline
        \multicolumn{6}{c}{\bf Transfer factors (in \%)} \\ \hline
        ISR/FSR &    &    $3.2$&          $28$&   $5$&    $1.1$\\
        PS &   &     $1$&    $62$&   $160$&          $6$\\
        Generator (aMC@NLO) &  &      $57$&   $160$&          $340$&          $38$\\
        Generator (Sherpa) &   &     $15$&   $90$&   $130$&          $6$\\        
        \hline       
        \end{tabular}
    \end{center}
    \caption{Theory uncertainties for the $\ttbar$\ background for the SRD regions.}
    \label{tab:ttbar_unc_SRD}
  \end{table} 

 \begin{table}[!h]
    \begin{center} \footnotesize
      \begin{tabular}{|c|c|c|c|}
        \hline
        & CRTopE & SRE & VRTopE\\
        \hline
        ttbar&   $42.5\pm 2.4 $&         $0.45\pm 0.21 $&        $66.5\pm 2.9 $\\
        ttbar (rad up)&          $53.4\pm 2.7 $&         $0.53\pm 0.19 $&        $84.1\pm 3.5 $\\
        ttbar (rad down)&        $35.4\pm 2.2 $&         $0.6\pm 0.4 $&          $60\pm 4 $\\
        ttbar (Powheg+H++)&      $36.4\pm 1.9 $&         $0.38\pm 0.20 $&        $72.6\pm 3.0 $\\
        ttbar (aMC@NLO+P8)&      $34\pm 11 $&    $<0.01$&        $44\pm 13 $\\
        ttbar (Sherpa)&          $45\pm 8 $&     $0.6\pm 0.6 $&          $74\pm 10 $\\
        \hline
        \multicolumn{4}{c}{\bf Transfer factors (in \%)} \\ \hline
        ISR/FSR &    &    $30$&   $4$\\
        PS &   &     $0$&   $27$\\
        Generator (aMC@NLO) & & $100$ & $44$\\
        Generator (Sherpa) & & $30$ & $5$\\        
        \hline       
        \end{tabular}
    \end{center}
    \caption{Theory uncertainties for the $\ttbar$\ background for the SRE regions.}
    \label{tab:ttbar_unc_SRE}
  \end{table}
 
 
%  \begin{table}[!h]
%    \begin{center} \footnotesize
     %\begin{tabular}{|c|c|c|c|c|c|c|c|c|c|c|}
     %    \hline
     %    & SRA-TT & SRA-TW & SRA-T0 & SRB-TT & SRB-TW & SRB-T0 & SRB-Other\\ \hline
     %   ISR/FSR &      $53$&   $26$&   $12$&   $41$&   $22.5$&         $7.2$&          $14.9$\\       
     %   PS  &  $37$&   $19$&   $7$&    $18$&   $8$&    $11.9$&         $2.2$\\
     %   Generator &  $80$&   $10$&   $40$&   $26$&   $45$&   $7$&    $13$\\
     %    \hline \hline
     % \end{tabular}
%    \end{center}
%    \caption{Summary of the \ttbar\ theory uncertainties obtained in the SRA signal regions. }
%    \label{tab:ttbar_unc}
%  \end{table}         


%   \begin{table}[!h]
%    \begin{center} \footnotesize
     %  \begin{tabular}{c||c|c|c|c|c|} \hline\hline
%         & SRC-low & SRC-med & SRC-high & SRE & SRF\\ \hline 
%          ISR/FSR &         $27$&   $59$&   $41$&   $25$&   $22$\\
%          PS &   $52$&   $56$&   $17$&   $16$&   $24$\\
% %         Generator & -\\
%         \hline \hline 
%      \end{tabular}
%    \end{center}
%    \caption{Summary of the $t\bar{t}$ theory uncertainties obtained in the SRC, SRE and SRF signal regions. The uncertainties are symmetrised, and all numbers are given in percentages. The MC statistical uncertainties are too large to meaningfully evaluate the generator uncertainty.}
%    \label{tab:ttbar_unc2}
%  \end{table}

%   \begin{table}[!h]
%    \begin{center} \footnotesize
      % \begin{tabular}{|c|c|c|c|c|c|c|c|c|}
     %    \hline
     %    & SRD1 & SRD2 & SRD3 & SRD3 & SRD4 & SRD6 & SRD7 & SRD8\\ \hline 
     %    ISR/FSR &  $15$&   $14$&   $14.4$&         $13$&   $10.7$&         $9$&    $8$&    $9$\\
     %    PS  &  $3$&    $6$&    $1$&    $1$&    $6$&    $1$&    $8$&    $14$\\
     %    Generator &   $18$&   $30$&   $19$&   $23$&   $11$&   $31$&   $34$&   $0$\\
     %    \hline \hline 
     % \end{tabular}
%    \end{center}
%%    \caption{Summary of the $t\bar{t}$ theory uncertainties obtained in the SRD signal regions. The uncertainties are symmetrised, and all numbers are given in percentages. }
%    \label{tab:single_top_unc3}
%  \end{table}

\item[\boldmath $\ttV$] The \ttV\ theory uncertainties on yields of $\ttV$ have been determined by performing the analysis with several variations at particle level. The particle-level object definitions are the same as for $\ttbar$. The systematic uncertainty is caculated as the difference in yields in due to the variations. %{\color{red} This depends on whether we end up using ttGamma.}

  Following the recommendations of the background forum, a global 13\% uncertainty on the production cross section and, therefore, on the global normalisation of the MC samples is considered. On top of this, further uncertainties on the acceptance of the selection have been considered:

  \begin{itemize}
  \item{\bf Scale} Scale variations are used to estimate the uncertainty on the ttV yield due to various scales. 
  \item{\bf Generator} Samples produced with the \sherpa\ generator are compared with the default MadGraph generated samples.
  \end{itemize}
  Table1~\ref{tab:ttbarZ_unc1} shows the estimated systematic uncertainties for all variations. Also mentioned earlier is the 13\% uncertainty on the total cross section, together with the total estimated uncertainty, obtained as the sum in quadrature of the individual uncertainties.

  %% ADDED BY FABRIZIO 10/2/2017
  The largest uncertainties are 35\% and 24\% in SRC1 and SRC5, respectively. This is expected due to the low number of $\ttV$ events falling in such regions. However, in the regions where such background is more significant, such as SRA, SRB, and SRD, the largest uncertainties are 5.2\%, 5.0\%, and 6.5\%, respectively.
  %%
  
  The $\ttV$ theory uncertainty is denoted by {\bf theoSysTtbarV} in the fit.

  % \begin{table}[!h]
  %   \begin{center} \footnotesize
  %         %           \begin{tabular}{c|c|c|c|c|c|c|c}   
      
  %         %         \end{tabular}
  %   \end{center}
  %   \caption{Summary of the \ttW\ theory uncertainties obtained in each of the signal regions. No number is reported for the SRD, because all uncertainties on acceptance are negligible. All numbers are in percentage.}
  %   \label{tab:ttbarW_unc}
  % \end{table}         

  % \begin{table}[!h]
  %   \begin{center} \footnotesize
  %         %           \begin{tabular}{c|c|c|c|c|c|c|c}   
  %         %         \end{tabular}
  %   \end{center}
  %   \caption{Summary of the \ttZ\ theory uncertainties obtained in each of the 6-jet signal regions. No number is reported for the SRD, because all uncertainties on acceptance are negligible. All numbers are in percentage.}
  %   \label{tab:ttbarZ_unc}
  % \end{table}
  


  %% COMMENTED OUT BY FABRIZIO 10/2/2017
  %% \begin{table}[!h]
  %%   \begin{center} \footnotesize
  %%     \begin{tabular}{|c|c|c|c|c|c|c|c|c|c|c|} 
  %%       \hline
  %%       & SRA-TT & SRA-TW & SRA-T0 & SRB-TT & SRB-TW & SRB-T0 & SRB-Other\\ \hline
  %%       Scale &  $5.8$&          $3.2$&          $2.9$&          $1.2$&          $0.0$&          $0.5$&          $5.0$\\
  %%       Generator &   $20.6$&         $25.8$&         $42.3$&         $17.7$&         $20.8$&         $18.0$&         $21.9$\\
  %%       \hline \hline 
  %%    \end{tabular}
  %%   \end{center}
  %%   \caption{Summary of the $t\bar{t}Z$ theory uncertainties obtained in the SRA and SRB signal regions. The uncertainties are symmetrised, and all numbers are given in percentages. }
  %%   \label{tab:ttbarZ_unc1}
  %% \end{table}



  
  %% ADDED BY FABRIZIO 10/2/2017
  \begin{table}[!h]
    \begin{center} \footnotesize
      \begin{tabular}{c||c} \hline\hline
{\bf SR} & {\bf uncertainty (\%)} \\ \hline
SRA-TT & 15.1\\ \hline
SRA-TW & 9.9\\ \hline
SRA-T0 & 13.7\\ \hline
SRB-TT & 7.3\\ \hline
SRB-TW & 5.7\\ \hline
SRB-T0 & 3.5\\ \hline
SRC1   & 95.5\\ \hline
SRC2   & 20.6\\ \hline
SRC3   & 21.4\\ \hline
SRC4   & 36.6\\ \hline
SRC5   & 30.9\\ \hline
SRD-low & 12.3\\ \hline
SRD-high & 15.1\\ \hline
SRE & 55.0\\ \hline
\hline
\end{tabular}

    \end{center}
    \caption{Summary of the theory uncertainties (in percent) on  $\ttV$ production obtained on the transfer factor on all the signal regions. The uncertainties are symmetrised, and all numbers are given in percentages.}
    \label{tab:ttbarZ_unc1}
  \end{table}
  %%
  


  
  %%  \begin{table}[!h]
  %%   \begin{center} \footnotesize
  %%    %  \begin{tabular}{c||c|c|c|c|c|} \hline\hline
  %%    %    & SRC-low & SRC-med & SRC-high & SRE & SRF\\ \hline 
  %%    %    Scale &           $9.6$&          $15.6$&         $6.1$&          $7.9$&          $10.2$\\
  %%    %    Generator &          $29.9$&         $39.3$&         $36.8$&         $17.7$&         $26.9$\\
  %%    %    \hline \hline 
  %%    % \end{tabular}
  %%   \end{center}
  %%   \caption{Summary of the $t\bar{t}Z$ theory uncertainties obtained in the SRC, SRE and SRF signal regions. The uncertainties are symmetrised, and all numbers are given in percentages. }
  %%   \label{tab:ttbarZ_unc2}
  %% \end{table}

  %%  \begin{table}[!h]
  %%   \begin{center} \footnotesize
  %%    %  \begin{tabular}{|c|c|c|c|c|c|c|c|c|}
  %%    %    \hline
  %%    %    & SRD1 & SRD2 & SRD3 & SRD3 & SRD4 & SRD6 & SRD7 & SRD8\\ \hline 
  %%    %    Scale &   $17.3$&         $3.3$&          $8.7$&          $9.0$&          $11.0$&         $5.6$&          $4.9$&          $15.0$\\
  %%    %    Generator &   $29.6$&         $17.2$&         $15.9$&         $37.7$&         $0.8$&          $14.5$&         $2.0$&          $7.7$\\
  %%    %    \hline \hline 
  %%    % \end{tabular}
  %%   \end{center}
  %%   \caption{Summary of the $t\bar{t}Z$ uncertainties obtained in each of the SRD signal regions. The uncertainties are symmetrised, and all numbers are given in percentages. }
  %%   \label{tab:ttbarZ_unc3}
  %% \end{table}
  

\item[Single-top] The single-top background in the signal region is dominated by the $Wt$ subprocess. The following comparisons are made to evaluate the uncertainty on the transfer factor. 

  \begin{itemize}
  \item {\bf Parton shower:} An uncertainty on the choice of the showering model is computed by comparing the baseline DSID 410013, 407014 (that is, \powheg\pythia\ with the P2011 tuning for \pythia, where the two DSIDs correspond to the top and antitop samples) and DSID 410147, 410148 (that is, \powheg\herwigpp\ with the UE5C6L1 CT10 tuning for \herwigpp\ and similar \met\ slicing). In this case, the uncertainty is given by equation~\ref{eq:ttbar_frag_uncertainty}.

  \item {\bf ISR/FSR:} Uncertainties related ISR/FSR are evaluated by comparing the radHi and radLow \powheg\pythia\ samples with DSID 410099, 410101 and 410100, 410102, where the two sets of DSIDs represent the top and antitop final states. The formula used is equation~\ref{eq:theory_uncertainty}.
  \end{itemize}

  The uncertainties resulting from these comparisons are summarized in Tables \ref{tab:single_top_unc1}, \ref{tab:single_top_unc2}, \ref{tab:single_top_unc3}, \ref{tab:single_top_unc1}, for SRA, SRB, SRC, SRD, SRE and associated control regions, respectively. 

  The single-top theory uncertainty is denoted by {\bf theoSysSingleTop} in the fit.

  \begin{table}[!h]
    \begin{center} \footnotesize
   \begin{tabular}{|c|c|c|c|c|c|c|c|} 
        \hline
        & CRST & CRTopATT & CRTopATW & CRTopAT0 & SRA-TT & SRA-TW & SRA-T0 \\ \hline
        \hline \hline 
        st Wt (MET200)&          $41.7\pm 1.1$&          $6.0\pm 0.4$&   $3.94\pm 0.35$&         $3.17\pm 0.31$&         $3.35\pm 0.33$&         $2.39\pm 0.27$&         $4.3\pm 0.4$\\
st Wt (radHi, MET200)&   $50.4\pm 1.3$&          $8.0\pm 0.5$&   $5.0\pm 0.4$&   $4.1\pm 0.4$&   $5.5\pm 0.4$&   $4.4\pm 0.4$&   $5.7\pm 0.4$\\
st Wt (radLo, MET200)&   $34.9\pm 1.0$&          $4.7\pm 0.4$&   $3.75\pm 0.33$&         $2.56\pm 0.28$&         $2.25\pm 0.25$&         $1.99\pm 0.24$&         $2.87\pm 0.29$\\
st Wt (Powheg+H++,MET200)&       $39.2\pm 1.0$&          $5.6\pm 0.4$&   $3.15\pm 0.30$&         $2.39\pm 0.27$&         $2.77\pm 0.28$&         $2.58\pm 0.27$&         $4.02\pm 0.34$\\
st Wt (DS,MET200)&       $6.8\pm 0.4$&   $0.81\pm 0.13$&         $1.42\pm 0.17$&         $1.14\pm 0.16$&         $0.059\pm 0.034$&       $0.08\pm 0.04$&         $0.06\pm 0.06$\\
       \hline
       \multicolumn{8}{c}{\bf Transfer factors (in \%)} \\ \hline
       \hline
       ISR/FSR &  &       $8\pm6$&        $4\pm6$&        $5\pm8$&        $26\pm7$&       $21\pm8$&       $16\pm6$\\
       PS &    &    $1\pm10$&       $15\pm13$&      $20\pm14$&      $12\pm14$&      $15\pm17$&      $1\pm13$\\
       Inteference (DR vs DS) & &        $17\pm16$&      $121\pm33$&     $120\pm40$&     $89\pm15$&      $79\pm18$&      $91\pm16$\\
       \hline
   \end{tabular}
    \end{center}
    \caption{Summary of the single-top theory uncertainties obtained in each of the signal regions. The uncertainties are symmetrised, and all numbers are given in percentages. Note that the event yields and transfer factors are obtained from truth-level samples. %The total uncertainty is the sum in quadrature of the uncertainties labeled ``Total    generator+PS+interference''. 
}
    \label{tab:single_top_unc1}
 \end{table}

  \begin{table}[!h]
    \begin{center} \footnotesize
   \begin{tabular}{|c|c|c|c|c|c|c|c|c|c|c|} 
        \hline
        & CRST & CRTopBTT & CRTopBTW & CRTopBT0 & SRB-TT & SRB-TW & SRB-T0 \\ \hline
        \hline \hline 
       st Wt (MET200)&          $41.7\pm 1.1$&          $7.3\pm 0.5$&   $7.9\pm 0.5$&   $12.0\pm 0.6$&          $11.4\pm 0.6$&          $15.1\pm 0.7$&          $53.8\pm 1.3$\\
st Wt (radHi, MET200)&   $50.4\pm 1.3$&          $9.6\pm 0.6$&   $9.8\pm 0.6$&   $13.4\pm 0.7$&          $15.4\pm 0.7$&          $19.8\pm 0.8$&          $59.9\pm 1.4$\\
st Wt (radLo, MET200)&   $34.9\pm 1.0$&          $5.8\pm 0.4$&   $7.7\pm 0.5$&   $10.5\pm 0.6$&          $8.8\pm 0.5$&   $11.5\pm 0.6$&          $48.5\pm 1.2$\\
st Wt (Powheg+H++,MET200)&       $39.2\pm 1.0$&          $6.5\pm 0.4$&   $7.1\pm 0.4$&   $9.3\pm 0.5$&   $10.7\pm 0.5$&          $14.7\pm 0.6$&          $44.9\pm 1.1$\\
st Wt (DS,MET200)&       $6.8\pm 0.4$&   $0.90\pm 0.14$&         $3.04\pm 0.25$&         $5.70\pm 0.35$&         $0.67\pm 0.12$&         $1.73\pm 0.20$&         $9.0\pm 0.5$\\
        \hline
        \multicolumn{8}{c}{\bf Transfer factors (in \%)} \\ \hline        
        ISR/FSR &   &     $7\pm5$&        $6\pm5$&        $6\pm4$&        $10\pm4$&       $9\pm4$&        $7.8\pm2.6$\\
        PS &    &    $5\pm10$&       $4\pm9$&        $18\pm8$&       $0\pm8$&        $4\pm7$&        $11\pm5$\\
        Interference (DR vs DS) &   &     $24\pm15$&      $136\pm27$&     $191\pm28$&     $64\pm10$&      $30\pm11$&      $3\pm9$\\
       \hline
   \end{tabular}
    \end{center}
    \caption{Summary of the single-top theory uncertainties obtained in each of the signal regions. The uncertainties are symmetrised, and all numbers are given in percentages. Note that the event yields and transfer factors are obtained from truth-level samples. %The total uncertainty is the sum in quadrature of the uncertainties labeled ``Total    generator+PS+interference''. 
}
    \label{tab:single_top_unc2}
  \end{table}

   \begin{table}[!h]
    \begin{center} \footnotesize
        \begin{tabular}{|c|c|c|c|c|c|c|c|}
        \hline
        & CRST  & SRC1 & SRC2 & SRC3 & SRC4 & SRC5 & VRTopC\\ \hline
          \hline
          st Wt (MET200)&          $41.7\pm 1.1$&          $0.66\pm 0.14$&         $1.14\pm 0.18$&         $0.99\pm 0.17$&         $0.39\pm 0.11$&         $0.12\pm 0.06$&         $19.9\pm 0.8$\\
          st Wt (radHi, MET200)&   $50.4\pm 1.3$&          $0.60\pm 0.14$&         $1.26\pm 0.20$&         $1.33\pm 0.21$&         $0.57\pm 0.14$&         $0.25\pm 0.09$&         $21.9\pm 0.8$\\
st Wt (radLo, MET200)&   $34.9\pm 1.0$&          $0.57\pm 0.13$&         $0.77\pm 0.15$&         $0.77\pm 0.15$&         $0.37\pm 0.10$&         $0.09\pm 0.05$&         $16.9\pm 0.7$\\
st Wt (Powheg+H++,MET200)&       $39.2\pm 1.0$&          $0.62\pm 0.13$&         $0.84\pm 0.16$&         $0.79\pm 0.15$&         $0.38\pm 0.10$&         $0.08\pm 0.05$&         $18.7\pm 0.7$\\
st Wt (DS,MET200)&       $6.8\pm 0.4$&   $0.12\pm 0.05$&         $0.30\pm 0.09$&         $0.23\pm 0.08$&         $0.16\pm 0.06$&         $0.020\pm 0.020$&       $4.39\pm 0.31$\\
          \hline \hline 
          \multicolumn{8}{c}{\bf Transfer factors (in \%)} \\ \hline
          ISR/FSR& &       $16\pm17$&      $6\pm13$&       $9\pm13$&       $3\pm18$&       $32\pm32$&      $5.4\pm3.4$\\
          PS &   &     $0\pm30$&       $22\pm22$&      $15\pm24$&      $0\pm40$&       $30\pm70$&      $0\pm7$\\
          Interference (DR vs DS) &  &      $10\pm50$&      $60\pm50$&      $40\pm50$&      $150\pm110$&    $0\pm110$&      $35\pm13$\\
          \hline
        \end{tabular}
    \end{center}
    \caption{Summary of the single-top theory uncertainties obtained in each of the signal regions. The uncertainties are symmetrised, and all numbers are given in percentages. Note that the event yields and transfer factors are obtained from truth-level samples. %The total uncertainty is the sum in quadrature of the uncertainties labeled ``Total    generator+PS+interference''. 
}
    \label{tab:single_top_unc3}
  \end{table}


  \begin{table}[!h]
    \begin{center} \footnotesize
      \begin{tabular}{|c|c|c|c|c|c|}
        \hline
        & CRST & CRW & SRD-low & SRD-high & VRTopD \\ \hline
        \hline \hline
        
       \hline
       \multicolumn{6}{c}{\bf Transfer factors (in \%)} \\ \hline
       \hline

      \end{tabular}
    \end{center}
    \caption{Summary of the single-top theory uncertainties obtained in each of the signal regions. The uncertainties are symmetrised, and all numbers are given in percentages. Note that the event yields and transfer factors are obtained from truth-level samples. %The total uncertainty is the sum in quadrature of the uncertainties labeled ``Total    generator+PS+interference''. 
}
    \label{tab:single_top_unc4}
 \end{table}

  \begin{table}[!h]
    \begin{center} \footnotesize
      \begin{tabular}{|c|c|c|c|}
        \hline
        & CRST & SRE & VRTopE \\ \hline
        \hline \hline
        st Wt (MET200)&          $41.7\pm 1.1$&          $0.93\pm 0.17$&         $2.81\pm 0.29$\\
        st Wt (radHi, MET200)&   $50.4\pm 1.3$&          $1.77\pm 0.24$&         $3.57\pm 0.34$\\
        st Wt (radLo, MET200)&   $34.9\pm 1.0$&          $0.37\pm 0.10$&         $2.10\pm 0.25$\\
        st Wt (Powheg+H++,MET200)&       $39.2\pm 1.0$&          $0.73\pm 0.15$&         $2.77\pm 0.27$\\
        st Wt (DS,MET200)&       $6.8\pm 0.4$&   $<0.01$&        $0.89\pm 0.14$\\
       \hline
       \multicolumn{4}{c}{\bf Transfer factors (in \%)} \\ \hline
       \hline
       ISR/FSR &  &      $54\pm14$&      $8\pm8$\\
       PS &     &   $16\pm25$&      $5\pm15$\\
       Interference (DR vs DS) &    &    $100\pm26$&     $90\pm40$\\
       \hline
      \end{tabular}
    \end{center}
    \caption{Summary of the single-top theory uncertainties obtained in each of the signal regions. The uncertainties are symmetrised, and all numbers are given in percentages. Note that the event yields and transfer factors are obtained from truth-level samples. %The total uncertainty is the sum in quadrature of the uncertainties labeled ``Total    generator+PS+interference''. 
}
    \label{tab:single_top_unc5}
 \end{table}
  
A 100\% uncertainty due to interference is applied for single top.

\item[The Remaining Background Components]

A 50\% uncertainty is used for the dibosons estimate.
\item[Signal Component] 

  {\color{red} Coming soon}

\end{description}
