\section{Inner Detector Track Reconstruction}
\label{sec:reco:IDtrack}

\indent  Many reconstructed physics objects depend on tracking information in the inner detector (ID).  ID tracks are combined with the EM calorimeter and muon spectrometer information to identify and measure the momentum of electrons and muons.  Hadronic jets use ID tracks to determine if the jet originated from a heavy flavored hadron containing b-quarks or only light flavored hadrons.  ID tracks are also crucial to identifying whether objects originate from the interesting hard scattering interaction or a less interesting pile-up interaction.\\

\indent Two types of inner detector tracks are reconstructed, primary tracks and secondary tracks.  Primary tracks originate from the interaction point (IP) and are meant to reconstruct the trajectories of charged particles originating directly from the proton-proton collisions.  Secondary tracks target charged particles originating in the ID from secondary decays and interactions such as $\gamma \rightarrow e^{+}e^{-}$ conversions.  \\

\indent Primary tracks are reconstructed using the {\tt NEWT} algorithm.  The {\tt NEWT} algorithm starts from seed segments in the inner silicon detectors and extended outwards towards the TRT. Greater details can be found in reference [\cite{NEWT}]. A brief summary of the track reconstruction algorithm will be given here.\\ 

\indent First, seed segments are created from three space point measurements in silicon detectors.  Each pixel cluster corresponds to a single space point measurement since a single pixel sensor provides 3D position information.  Two SCT clusters from the same SCT layer must be combined to form a single space point measurement because each SCT strip only provides 2D position information.  \\

\indent Seed segments can be formed out of all pixel (PPP), all SCT (SSS) or two pixel and one SCT (PPS) space points.  PSS space points are rejected due to high fake rates.  \\

\indent Starting from the original seed segment, track reconstruction is extrapolated layer by layer through the inner detector. Hits are added to the track one layer at a time.  If multiple tracks share a hit, then the shared hit is assigned to the most precise track.  \\%If multiple charged particles are close together when they traverse a pixel layer, they can form merged clusters in the Pixel detector.  These merged pixel clusters are split using a set of trained neural networks. \\

\indent In contrast, secondary tracks are reconstructed from the TRT and extrapolated inwards towards the direction of the beam line.  Segments are reconstructed in the TRT and then extended inwards by adding silicon hits. \\