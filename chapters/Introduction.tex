%!TEX root = ../dissertation.tex
\chapter{Introduction}
\label{introduction}

\indent The Standard Model of Elementary Particles (SM) provides a concrete description of the interactions and dynamics of all known elementary particles with the exception of gravity.  In the SM, matter is composed of three generation of fermions with spin 1/2 while interactions are governed by gauge symmetries and mediated by spin 1 gauge bosons.  \\

\indent The last piece of the SM, the Higgs boson, was discovered in 2012 at the Large Hadron Collider.  The complex scalar Higgs field spontaneously break the electroweak (EW) symmetry by acquiring a vacuum expectation value (VEV). This process of electroweak symmetry breaking (EWSB) gives mass to the $W$ and $Z$ gauge bosons. The fermions also acquire their mass through Yukawa interactions between the Higgs and fermion fields. \\

\indent Although the Brout-Englert-Higgs mechanism ensure that the SM theory will remain viable as a perturbative physical theory up to the Planck scale, current experimental evidence suggests that the SM is not a complete theory of nature.  SM leaves several important fundamental questions unanswered.  These open questions include but is not limited to the nature of Dark Matter (DM), the apparent matter/anti-matter asymmetry in the universe,  the reason behind the SM mass spectrum, the potential unification of EW and strong interactions at high energy scales, the nature of neutrino mass and the hierarchy problem regarding the naturalness of the Higgs mass.  The answers to these questions are at the frontier of physics research and form the major physics goals of many different physics experiments across multiple disciplines. \\

\indent One proposed solution to many of these questions is the introduction of a new spacetime symmetry called supersymmetry.  Supersymmetry imposes a new symmetry between fermions and bosons allowing one to transform into the other.  In this way, the supersymmetric extension to the SM (SUSY) predicts the existence of a yet undiscovered superpartner to every known SM particle.  SUSY gives one possible solution to the hierarchy problem of the Higgs as large contributions to the Higgs potential are canceled out between SM particles and their superpartners.  Some supersymmetric models also unify the strong and electroweak force at high energies, provides more cp violation to generate matter/antimatter asymmetry and produces plausible dark matter candidates.  Plus when SUSY is imposed as a local symmetry, general relativity is automatically included offering a potential path to uniting general relativity with quantum mechanics.  \\

\indent All previous high energy experiments including the Tevatron and LEP have not detected the existence of superpartners leading us to believe that SUSY is a spontaneously broken symmetry.  Many different SUSY symmetry breaking mechanism have been proposed but they all make the superpartners more massive then their SM counterparts. \\

\indent A major goal of the Large Hadron Collider (LHC) experiment is to search for the predicted superpartners at an unprecedented energy scale.  If SUSY is the solution to the hierarchy problem and restores naturalness to the Higgs mechanism then the superpartner to the top quark (stop) is expected to be no heavier then a few $\tev$.  The stop's mass is strongly constrained due to the large coupling between the SM top quark and the Higgs with $\lambda_t \sim 0.94$.  As such, searches for the stop at the LHC is especially interesting because the stop's mass may be low enough to be directly produced at the energy scale of the LHC. \\

\indent This thesis concerns the search for stops in an traditionally experimentally difficult region.  One expected stop decay channel produces a the top quark along with the superpartner to a neutral electroweak boson, the neutralino (\ninoone). Traditional searches for SUSY often targets the experimental signatures of neutralinos as they are unique to SUSY.  Experimentally this involves searching for events with large missing transverse energy ($\met$).  \\

\indent The traditional search strategy can effectively detect stops with large mass splittings between $m_{\stop}$ and $m_{\ninoone}$.  However, when the stop mass is nearly degenerate to $m_t + m_{ninoone}$ the stop has just enough energy to produce the top and neutralino.  The resulting stop decay produces gain little momenta from the stop decay. The low $\pt$ neutralinos in turn generate very little $\met$. The only other observables in the event are the visible tops which are also produced in SM top/anti-top pair production.  This inability to distinguish SM ttbar from stop signal greatly hampers the search sensitivity in this region.  For example, SM ttbar can have a production cross-section anywhere from 50 to 300 times that of the stop in our region of interest. \\

\indent The low decay product $\pt$ problem is ubiquitous to all regions of phase space with small mass splittings.  Many ATLAS searches in SUSY including charginos, Higgsinos, sbottom, sleptons, etc all have some region of phase space with a compressed mass spectra.  In general, such regions of phase space are called compressed regions. \\

\indent This thesis demonstrates a new method of searching for stops in the compressed region by isolating events with strong initial state radiation (ISR).  The ISR boosts the stops and gives additional momenta to the stop decay products.  The correlation between ISR pt and stop decay product $\pt$ tend to be extremely strong in this region precisely because the stop decay products gain little momenta from the stop decays.  Specifically there exists a strong correlation in both direction and magnitude between ISR $\pt$ and $\met$ in signal.  The neutralinos will inherit a fraction of the original ISR $\pt$ proportional to  $m_{\ninoone}/m_{\stop}$ and $\met$ and ISR systems should be back to back.  This two-dimensional correlation allows us to separate signal from ttbar background and overcome the difference in production cross-section. We effectively converted the critical weakness of the region, decay products gaining little $\pt$ from the stop decays, into a strength.  \\

\indent We also developed an accurate ISR identification system for this analysis.  The algorithm works by first finding the axis of maximum back to back $\pt$ call the thrust axis.  The thrust axis should mimic the axis of back to back boost between the ISR and sparticle systems in events with strong ISR because the ISR and sparticle boost represents the single largest back to back kick in events with strong ISR.  We then divide the event into two hemispheres according to the thrust axis.  All objects in the same hemisphere as the $\met$ are considered to have originated from a stop decay because we expect the neutralinos to travel in the same direction as the original stops.  All objects in the hemisphere opposite the $\met$ are considered to have originated from ISR.  In this way, the thrust based algorithm is able to identify entire ISR systems composed of multiple jets.  \\ 

\indent The ISR identification algorithm is completely general and can be used to identify ISR for SM processes as well as other BSM searches.  The performance of the ISR identification algorithm is covered in detail in chapter \ref{chap:jigsaw}.  In summary, the algorithm can achieve a 9 percent uncertainty on the reconstructed ISR $\pt$ in stop and ttbar events with at least 400 $\gev$ of true ISR $\pt$. This uncertainty includes any detector uncertainty due to the reconstruction of jets, $\met$ and other physics objects.  \\

\indent The ISR based search has allowed us to finally making a definitive statement on the existence of stops in a region with no previous exclusion sensitivity.  Using the $\intlumi$ $\ifb$ 2015 and 2016 $\sqrt{s}=13 \tev$ dataset, we were able to exclude stops in this compressed region with masses between 225 and 600 $\gev$ to 95 percent confidence.  We were able to achieve expected exclusion confidence limits of less then $5 \times 10^{-4}$ for stop masses between 250 and 400 $\gev$.   \\

\indent The methods demonstrated in this thesis can be applied to other compressed region searches and searches involving ISR such as dark matter searches.  The accurate ISR identification algorithm can also directly measure the amount of ISR produced in conjunction with SM particles. Potential applications includes measuring SM ttbar's ISR $\pt$ spectrum. \\

\indent The thesis is organized as follows.  Chapter \ref{chap:motivation} presents an overview of the standard model and theoretical motivations for supersymmetry.  Chapter \ref{chap:Exp} describes the experimental setup of the LHC accelerator and ATLAS detector.   Chapter \ref{chap:reconstruction} and chapter \ref{chap:trigger} details the reconstruction and calibration of physics objects at ATLAS the ATLAS trigger system.  \\

\indent The physics objects used in the analysis are defined in chapter \ref{chap:objects}.  The Monte Carlo simulations of stop signal and SM background are described in Chapter \ref{chap:MCSimulation}.  \\ 

\indent Chapter \ref{chap:AnaStrategy} concerns the general strategy used in traditional SUSY searches and the general strategy of using ISR to separate signal from background in compressed region analyses.  We present the new thrust based ISR identification algorithm in chapter \ref{chap:jigsaw}.  The algorithm is explained in context of a more general set of algorithms that uses exterminations to classify objects called recursive jigsaw reconstruction.  The performance of the ISR identification algorithm is also demonstrated on both signal and background. \\

\indent Chapter \ref{chap:data} - \ref{chap:SignalRegion} describes the 2015 and 2016 LHC dataset that is used for this analysis and the kinematic selections used to define the signal region (SR).  The chapters develop physical intuition on each signal region selection and explain how they reject different background.  \\

\indent The SM backgrounds in the SR are described in detail in chapter \ref{chap:backgrounds}.  This chapter covers the different methods used to estimate each background, going especially into detail on the dominate background: SM ttbar.  A large portion is devoted to building intuition on the unique kinematic properties of each background.  This physical intuition is used to explain the CR design and how the CRs are able to accurately estimate the background rate and minimize systematic uncertainties.  \\

\indent Chapter \ref{chap:Uncertainties} describe each of the experimental and theoretical systematics associated with signal and background.  Systematic uncertainty is divided into two categories; experimental uncertainties due to limitations on detector resolution and theoretical uncertainties on the Monte Carlo simulations. \\

\indent Chapter \ref{chap:statistics} summarizes the statistical methods used to extract the signal strength.  Finally chapter \ref{chap:Results} and \ref{chap:Interpretation} show the results with $\intlumi$ $\ifb$ of $\sqrt{s} = 13 \tev$ data and give an interpretation of the results on select signal models.  \\


%Supersymmetry adds four fermionic coordinates in addition to the usual four bosonic coordinates $(t,x,y,z)$

%Supersymmetry forms an extension to the usual space-time coordinates of $(t,x,y,z)$ and allows for transformations between fermion and bosons.  