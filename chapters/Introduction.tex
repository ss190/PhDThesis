%!TEX root = ../dissertation.tex
\chapter{Introduction}
\label{introduction}

\indent The Standard Model of Elementary Particles (SM) provides a concrete description of the dynamics and interactions of all known elementary particles with the exception of gravity.  In the SM, matter is composed of three generation of fermions with spin 1/2 while interactions are governed by gauge symmetries and mediated by spin 1 gauge bosons.  \\

\indent The last piece of the SM, the Higgs boson was discovered in 2012 at the large hadron collider.  The complex scalar Higgs field spontaneously break the electroweak (EW) symmetry by acquiring a vacuum expectation value (VEV). This process of electroweak symmetry breaking (EWSB) gives mass to the $W$ and $Z$ gauge bosons. The fermions also acquire their mass through Yukawa interactions between the Higgs and fermion fields. \\

\indent Although the Brout-Englert-Higgs mechanism ensure that the SM theory will remain viable as a perturbative physical theory up to the Planck scale, current experimental evidence suggests that the SM is not a complete theory of nature.  SM leaves several important fundamental questions unanswered.  These include but is not limited to the nature of Dark Matter (DM), the apparent matter/anti-matter asymmetry in the universe,  why does SM particles have the mass spectrum that they do, the potential unification of EW and strong interactions at high energy scales, the nature of neutrino mass and the hierarchy problem regarding the naturalness of the Higgs mass.  The answers to these open questions are the major physics goals of many different physics experiments across disciplines including the energy frontier, intensity frontier and experiments in astrophysics. \\

\indent One proposed solution to many of these questions is the introduction of a new spacetime symmetry called supersymmetry.  Supersymmetry imposes a new symmetry between fermions and bosons allowing one to transform into the other and a supersymmetric Lagrangian is invariant under such transformations.  In this way, the supersymmetric extension to the SM (SUSY) predicts the existence of complete set of new particles composed of a yet undiscovered superpartner to every known SM particle.  SUSY is one possible solution to the hierarchy problem of the Higgs as large contributions to the Higgs potential are canceled between particles and their superpartners at high mass scales.  Some supersymmetric models also unify the strong and electroweak force at high energies, generates more cp violation necessary to generate the matter/antimatter asymmetry and produces plausible dark matter candidates.  Plus when SUSY is imposed as a local symmetry, general relativity (GR) is automatically included.  The resulting theories are referred to as supergravity and offers a potential to uniting GR with quantum mechanics.  \\

\indent SUSY is expected to be a spontaneously broken symmetry.  Many different SUSY symmetry breaking mechanism have been proposed but they all make the superpartners much more massive then their SM counterparts and explains why we have yet to see any superpartners in previous high energy experiments including the Tevatron and LEP.  \\

\indent A major goal of the Large Hadron Collider (LHC) experiment is to search for the predicted superpartners at an unprecedented energy scheme.  If SUSY is the solution to the hierarchy problem and restores naturalness to the Higgs mass calculation then the superpartner to the top quark (stop) is expected to be no heavier then a few $\tev$.  The stop's mass is strongly constrained due to the large coupling between the SM top quark and the Higgs ($\lambda_t \sim 0.94$).  As such, searches for the stop at the LHC is especially interesting because the stop's mass may be low enough to be accessible at the energy scale of the LHC. \\

\indent This thesis concerns the search for stops in an traditionally experimentally difficult region.  One of the decay mode of the stop is to decay into its partner, the top quark, and the superpartner to the neutral electroweak boson, the neutralino. Traditional searches often targets events with large missing transverse energy ($\met$), the experimental signatures of neutralinos because such large amounts of $\met$ is difficult to produce with SM processes alone.  However, when the stop mass is close to that of the top mass plus the neutralino mass, both the top and neutralino gain very little momenta from the decay. The invisible neutralinos in turn generate very little $\met$. This leaves only the visible tops which are mimicked by standard model ttbar.  The fact that SM ttbar can have a production cross-section anywhere from 50 to 300 times that of the stop even if the stop mass is low, around 250 to 400 $\gev$. means that searches for stops in this diagonal region have been hampered due to the difficulty of distinguishing between stop signal and ttbar background in this region of phase space. \\

\indent This thesis demonstrates a new method of searching for stops in the compressed region by isolating events with strong initial state radiation (ISR).  The stop decay products can gain additional momenta if the entire stop system is boosted by ISR.  The correlation between ISR pt and stop decay product $\pt$ tend to be extremely strong in this region precisely because the stop decay products gain little momenta from the stop decays in compress region.  we turn a critical weakness inherent to this region of phase space into a strength.  \\

\indent Specifically there exists a strong correlation in both direction and magnitude between ISR $\pt$ and $\met$ in signal.  The neutralinos should inherit exactly $m_{\ninoone}/m_{\stop}$ fraction of the original ISR $\pt$ if the stop has exactly enough mass to create a top and neutralino at rest and the $\met$ and ISR systems should be exactly back to back.  This two-dimensional correlation allows us to separate signal from ttbar background and overcome the difference in production cross-section. \\

\indent As part of this analysis we developed an accurate ISR identification system that is able to identify whole ISR systems composed of multiple jets.  The algorithm works by first finding the axis of maximum back to back $\pt$ call the thrust axis.  The ISR and sparticle boost represents the single largest back to back kick in events with strong ISR.  Therefore, the thrust axis should mimic the axis of back to back boost between the ISR and sparticle systems in events with strong ISR.  Once we find the thrust axis, we divide the event into an ISR hemisphere and an stop hemisphere according to the thrust axis.  All objects in the hemisphere contains the $\met$ is considered to have originated from a stop decay as the neutralinos must travel in the same direction as the original stops and all objects in the hemisphere opposite the $\met$ is considered to have originated from ISR. \\ 

\indent The ISR identification algorithm is completely general and can be used to identify ISR for SM processes as well as other searches.  The performance of this ISR identification algorithm is covered in detail in chapter \ref{chap:jigsaw}.  In summary, we see that we can achieve a 9 percent uncertainty on the reconstructed ISR pt for both stop and ttbar events with at least 400 $\gev$ of true ISR $\pt$. This uncertainty includes any detector uncertainty due to the reconstruction of jets, $\met$ and other physics objects.  \\

\indent Using the 2015 and 2016 $\sqrt{s}=13 \tev$ dataset, we were able to exclude stops with masses between 225 and 600 $\gev$.  In between 250 and 400, we are able to achieve expected exclusion p-values less then $5 \times 10^{-4}$ in a region with no previous sensitivity. \\

\indent Its important to note that the same problem of low $\pt$ decay products and low $\met$ arises in many other searches if the mass splittings are small or at a particular value where the decay products gains very little momenta. Such regions of phase space are referred to in general as "compressed regions" and occurs in many different searches for SUSY and other exotic particles.  The same problem and solution demonstrated in this thesis can be applied to searches to Higgsinos, Charginos, and even dark matter particles unrelated to SUSY.  Plus the accurate ISR identification algorithm can be used not only in searches for new physics but offers a new method for directly measuring the amount of ISR produced in conjunction with SM particles. Potential applications includes measuring SM ttbar's $tt~\pt$ without the need for reconstructing individual tops. \\

\indent The thesis is organized as follows.  Chapter \ref{chap:motivation} presents an overview of the standard model and theoretical motivations for supersymmetry.  Chapter \ref{chap:Exp} describes the experimental setup of the LHC accelerator and ATLAS detector.   Chapter \ref{chap:reco} and chapter \ref{chap:trigger} details the reconstruction and calibration of physics objects from detector signature and the ATLAS trigger system.  Physics objects used in the analysis is defined in chapter \ref{chap:obj}.  Chapter \ref{chap:simulation} describes the process used in generating the Monte Carlo simulations of stop signal and SM background processes.  \\ 

\indent Chapter \ref{chap:Ana} concerns general strategy used in traditional SUSY searches and the general strategy of using strong initial state radiation (ISR) to separate signal from background in compressed region analyses.  We present a new algorithms that uses the axis of maximum back to back $\pt$ to identify ISR in chapter \ref{chap:Jigsaw}.  The algorithm is based on a more general set of algorithms called recursive jigsaw reconstruction that uses exterminations to classify objects in events with missing information such as in the case of $\met$ and large multiplicity of objects.  The precision of the ISR identification algorithm is also demonstrated on both signal and background. \\

\indent Chapter \ref{chap:data} - \ref{chap:SignalRegion} describes the 2015 and 2016 LHC dataset that is used for this analysis and the kinematic selections used to define the signal region.  The chapters also develop physical intuition on each signal region selection and how they reject background.  \\

\indent The different SM backgrounds that remain in the signal region are described in detail in chapter \ref{chap:backgrounds}.  This chapter also cover the different methods used to estimate each background, goes especially into detail on the dominate background SM ttbar.  A large portion is devoted to building intuition on the unique kinematic properties of the backgrounds in the signal region.  This physical intuition is used to explain how the control regions are designed in order accurately estimate the backgrounds and minimize systematic uncertainties.  \\

\indent Chapter \ref{chap:systematics} describe each of the experimental and theoretical systematics associated with signal and background.  Systematic uncertainty is divided into two categories, experimental uncertainties due to limitations of detector resolution and object reconstruction algorithms and theoretical uncertainties on the calculation of Monte Carlo simulations. \\

\indent Chapter \ref{chap:statistics} summarizes the statistical methods used to extract the signal strength.  Finally chapter \ref{chap:results} and \ref{chap:interpretation} shows the result of the search and give the exclusion limits on stop parameter space.  \\


%Supersymmetry adds four fermionic coordinates in addition to the usual four bosonic coordinates $(t,x,y,z)$

%Supersymmetry forms an extension to the usual space-time coordinates of $(t,x,y,z)$ and allows for transformations between fermion and bosons.  