\chapter{Definitions of Reconstructed Physics Objects}
\label{chap:objects}

The object definitions used follow the ATLAS recommendations and SUSY group standards for MC15c samples ({\tt SUSYTools-00-08-33} and AnalysisBase-2.4.23), details of which are given below. %Currently the MET TST cleaning is not applied since we are using p2666 samples (following the recommendation).

\section{Electron Definition}  \label{sec:EleDef}

Electron candidates are selected using the {\tt VeryLooseLH} definition and are required to have $\et = E_{\mathrm{cl}} / \cosh{\eta} > 7 \gev$ and to fall within $|\eta_{\mathrm{cl}}| < 2.47$, where $E_{\mathrm{cl}}$ and $\eta_{\mathrm{cl}}$ are the energy and pseudorapidity of the electron candidate cluster, respectively~\cite{EgammaRun2Analyseswiki}. Currently, electrons falling in the so-called ``crack" region ($1.37 < |\eta| < 1.52$) and satisfying the {\tt VeryLooseLH} criteria are considered electron candidates.


Table~\ref{tb:electrons} summarizes the electron definition criteria and specifies the angular distance for which an electron overlapping with a jet is rejected (see section ~\ref{sec:Selection_overlap}).   


\begin{table}[htp]
  \caption{Electron selection criteria. The overlap removal is described in more detail in section \ref{sec:Selection_overlap}.}
  \begin{center}
    \begin{tabular}{c|c} \hline \hline
      Cut & Value \\ \hline \hline
      Acceptance & $\pT > 7 \gev$, $|\eta_{\mathrm{clust}}| < 2.47$ \\ \hline
      Quality & {\tt VeryLooseLH} \\ \hline
      Isolation & {\tt Gradient Loose} \\ \hline
      Overlap: $e$ rejected if &  $0.2 < \Delta R(e,jet) < 0.4$\\ \hline%, $\Delta R(e,jet)<0.2$ and jet is b-tagged \\ \hline
      \hline
    \end{tabular}
  \end{center}
  \label{tb:electrons}
\end{table}%


\section{Muon Definition}  \label{sec:MuonDef}
In order to be accepted, muon candidates must pass the {\tt Loose} selection criteria, have  $\pT > 6 \gev$, and fall within $\eta < 2.7$.
% , and pass the tracking quality cuts shown in Table~\ref{tb:muons}.  %Additionally, the \pT\ of muons in the MC are smeared. 

Table~\ref{tb:muons} summarizes the muon definition criteria and specify the angular distance for which a muon overlapping with a jet is rejected (see section ~\ref{sec:Selection_overlap}).\\     

\begin{table}[htp]
  \caption{Muon selection criteria. The overlap removal is described in more detail in section \ref{sec:Selection_overlap}.} 
  \begin{center}
    \begin{tabular}{c|c} \hline \hline
      Cut & Value \\ \hline \hline
      Acceptance & $\pT > 6\gev$, $|\eta| < 2.7$ \\ \hline
      Quality & Loose \\ \hline
      Overlap: $\mu$ is rejected if &  $ \Delta R(\mu,jet) < 0.4 $ \\ \hline
      \hline
    \end{tabular}
  \end{center}
  \label{tb:muons}
\end{table}%

\section{Jet Definitions}

\subsection{Calorimeter Jets}

Jets are reconstructed from topological clusters using the \antikt\ jet algorithm~\cite{antikt} with a distance parameter of $R = 0.4$ and full four-momentum recombination. The {\tt AntiKt4EMTopo} jet collection is used, and global sequential calibration is enabled. Area-based corrections are also applied. % and are calibrated using an area-based pileup subtraction with a local cluster weighting (LCW) algorithm ({\rm AntiKt4LCTopoJets}). The LCW algorithm identifies if a topological cluster in the calorimeter is of hadronic or electromagnetic origin and applies either the hadronic or electromagnetic energy correction as appropriate which relates the measured energy in the calorimeter to the actual amount of deposited energy.  The jet response is also dependent on the pileup conditions and this is accounted for using the jet area subtraction method provided by the JetEtMiss group~\cite{ATLAS:JetEtmissDataAnalysisRecommendationwiki}.  
All jets must have $\pT > 20\gev$ with no $\eta$ requirement to be considered in the analysis. Jets that pass this loose selection are considered when resolving overlapping objects and determining if the event should be vetoed due to bad quality jets.  After the overlap removal and jet quality cuts, jets are required to have $\pt > 20\gev$ and $|\eta|<2.8$ to be considered in the analysis. In addition to the $\eta$ and \pt\ requirement a jet vertex tagger~\cite{jvtTwiki} value greater than 0.59 is required, corresponding to a 92\% efficiency for jets originating from the hard scatter event and a 2\% fake rate from pileup jets, if the jet has $|\eta|<2.4$ and $\pt<60$ GeV. This requirement helps to reduce sensitivity to pileup jets.

% \subsection{Track Jets}

% Jets are reconstructed from charged tracks using the \antikt\ jet algorithm~\cite{antikt} with a distance parameter of $R = 0.4$ and full four-momentum recombination. The details of the track selection are under study for the 2015 analysis. {\color{red} Jets built from charged tracks are not currently available in the derived samples.}

\subsection{\boldmath $b$-tagged Jets}

A subset of the baseline jets in the event are identified as originating from the decay of a $b$-quark by requiring they pass the MV2c10 jet tagger and are within the tracker acceptance with $|\eta| < 2.5$.  The (MV2c10 $> 0.6459$) operating point is used, which corresponds to a $\sim77\%$ $b$-tagging efficiency. %~\cite{ATLAS-COM-CONF-2012-021}.
Events in the signal region must contain at least one $b$-tagged jet.  The $b$-tagging efficiencies and mistag rates have been measured by the Flavour Tagging group.  A correction scale factor, parameterized in  $\eta$, \pT\, and jet flavor, has been calculated by this group and applied to all jets in Monte Carlo samples~\cite{BtaggingCalibrationToolwiki}. 


\section{\met\ Definitions} 
\label{sec:Selection_MET}

\subsection{\boldmath Calorimeter-based \met}

The calorimeter-based \met\ is calculated from an object-based algorithm. The \met\ is recalculated with the object definitions above. Baseline muons, electrons, and jets after overlap removal are used in the \met\ recalculation performed within SUSYTools. An extra term is added to account for soft energy in the event that is not associated to any of the selected objects. This soft term is calculated from inner detector tracks with $\pT > 400 \MeV$ matched to the primary vertex to make it more resilient to pileup contaminations.

%Currently {\tt MET\_RefFinalFix} is used (but is not recalculated for the specific object definitions described above). 

% {\tt MET\_Egamma10NoTau\_RefFinal}~\cite{Aad:2012re}:
% \begin{equation}
%   {\met}^{\mathrm{RefFinal}} = {\met}^{\mathrm{RefEle}} + {\met}^{\mathrm{RefJet}} + {\met}^{\mathrm{RefMuon}} + {\met}^{\mathrm{CellOut}} + {\met}^{\mathrm{RefGamma}}.
%   \label{eqn:met}
% \end{equation}
% Muons passing the selection criteria and with $\pT > 10 \gev$ are included in the $ {\met}^{\mathrm{RefMuon}} $ term.  Topoclusters not assigned to reconstructed objects are included in the ${\met}^{\mathrm{CellOut}}$ term.

% The \met\ is then corrected for small differences between object definitions used in {\tt MET\_Egamma10NoTau\_RefFinal} and the SUSY group standard definitions outlined above (e.g. the smearing of the lepton \pT\ in the MC). This is done using the {\tt METUtility} tool.

\subsection{\boldmath Track-based \MET}

The \MET\ derived from the sum of the transverse momenta of the tracks in the event ($\mettrk = -\sum_i^{\mathrm{tracks}}\pT^i$). In the 2012 analysis, the azimuthal angle between this and the calorimeter-based \MET\ was an effective discriminant against events with fake \MET. The \mettrk\ is rebuilt using the muons, electrons, and jets, similar to the \met.

\section{Photon Definition}
\label{sec:PhoDef}
The photon reconstruction is only considered in the photon control region. Photon candidates are obtained and calibrated using {\tt GetPhoton} function in {\tt SUSYTools}. A signal photon must pass the selection requirements that are described in Table~\ref{tb:photons}.


\begin{table}[htp]
  \caption{Photon selection criteria.} 
  \begin{center}
    \begin{tabular}{c|c} \hline \hline
      Cut & Value \\ \hline \hline
      Acceptance & $\pT > 130 \gev$, $|\eta| < 2.37$ \\ \hline
      Quality & {\tt Tight} \\ \hline
      Isolation &  {\tt FixCutLoose} \\ \hline
      \hline
    \end{tabular}
  \end{center}
  \label{tb:photons}
\end{table}%

The \pt\ requirement is to ensure that the event is on the plateau of the photon trigger {\tt HLT\_g120\_loose}.

\section{Resolving overlapping objects}
\label{sec:Selection_overlap}

In the case of candidate objects overlapping with each other, all but one object must be removed from the event. The distance metric used to define overlapping objects is $\Delta R = \sqrt{\Delta \phi^2 + \Delta \eta^2}$. The baseline SUSYTools overlap removal provided by the SUSY group is applied with an additional overlap removal applied to objects that survive the baseline. The SUSYTools baseline overlap removal is: \\

\begin{itemize}
\item If an electron and jet are located with $\Delta R < 0.2$ consider the object an electron and remove the jet unless the jet is b-tagged (using the 85\% working point rather than the 77\% working point used in the signal regions) in which case the jet is kept and the electron is removed. 
\item If a muon and jet are located with $\Delta R < 0.4$, consider the object a jet and remove the muon. Unless the object has less than three tracks (with $\pt>500$ MeV), then the jet is removed while the muon is kept. 
\item If an electron and jet are located with $0.2 \leq\ \Delta R < 0.4$, consider the object a jet and remove the electron.
\end{itemize}

% Objects that pass the above requirements are tested for additional overlap requirements:
% \begin{description}
% \item [$\boldsymbol{e/\mu}$:] If a calo-tagged muon is within $\Delta R < 0.01$ of an electron, the muon is removed and the electron is kept
% \item [$\boldsymbol{e}$/jet:] If a jet is not b-tagged and is within $\Delta R<0.2$, the jet is removed and the electron is kept. 
% \item [$\boldsymbol{\mu}$/jet:] If a non-btagged jet is within $\Delta R < 0.2$ of a muon and either has less than three tracks (with $\pt>500$ MeV) or $\pt^{\mu} / \pt^{\textnormal{jet}}>0.7$, the jet is removed while the muon is kept. 
% \item [lepton/jet:] For lepton/jet overlap a variable $\Delta R$ requirement is used with a minimum and maximum truncation of 0.1 and 0.4, respectively. The variable $\Delta R$ requirement is a function of the lepton $\pt$: $\Delta R<0.04+10 GeV/\pt^{\ell}$. If a lepton is within the variable/truncated $\Delta R$ the lepton is removed and the jet is kept. 
% \end{description}
