%!TEX root = ../dissertation.tex
% the abstract

The ATLAS experiment at Large Hadron Collider (LHC) searches for experimental evidence of many new beyond the standard model physics at the TeV scale.   As we collect more data at the LHC we continue to extend our sensitivity to these new phenomenon, particularly probing increasingly more massive new particles.  Despite this progress there are still regions of parameter space where constraints remain weak.   One common cause of this lack of sensitivity is because the new particle has a very small mass splitting between it and its decay products.  The particle then has little energy left over to give momenta to its decay products and the low momenta decay products are difficult to experimentally detect. These regions of small mass splitting are called {\it compressed} regions.  We are able to gain sensitivity to these difficult regions by searching for new particles produced in conjunction with strong initial state radiation (ISR).  The strong initial state radiation boosts the new particle's decay products and gives them momentum.  \\

This thesis covers the search for the supersymmetric partner to the top quark (stop) in the region when the stop and its decay products are nearly degenerate in mass.  No searches prior to 2016 was sensitive to this region.  We were able to exclude stops up to a mass of 525 GeV in this region with the 2015 and 2016 ATLAS dataset.  I will demonstrate a new and more accurate technique for identifying whole initial state radiation systems instead of a single ISR jet.   As the LHC provides more data and traditional search methods rule out parameter space at higher masses, it becomes more important that we also gain sensitivity to these compressed regions that are still unconstrained at low masses.  I will show that this initial state radiation identification technique is completely generalizable and useful for many other searches that target small mass splittings.