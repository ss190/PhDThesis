%!TEX root = ../dissertation.tex
% the abstract

\indent The ATLAS experiment at Large Hadron Collider (LHC) searches for experimental evidence of many new beyond the standard model physics at the TeV scale.   As we collect more data at the LHC we continue to extend our sensitivity to these new phenomenon, particularly probing increasingly more massive new particles.  Despite this progress there are still regions of parameter space where constraints remain weak.   One common cause of this lack of sensitivity is because the new particle has a very small mass splitting between it and its decay products.  The particle then has little energy left over to give momenta to its decay products and the low momenta decay products are difficult to experimentally detect. These regions of small mass splitting are called compressed regions.  We are able to gain sensitivity to these difficult regions by searching for new particles produced in conjunction with strong initial state radiation (ISR).  The strong ISR boosts the new particle's decay products and gives them momentum.  \\

\indent This thesis covers the search for the supersymmetric partner to the top quark (stop) in the region when the stop and its decay products are nearly degenerate in mass.  No searches prior to 2016 were sensitive to this region.  We were able to exclude stops up to a mass of 600 GeV in this region with $\intlumi$ $\ifb$ of $\sqrt{s} = 13 \tev$ LHC proton-proton collision data.  This data was collected by the ATLAS experiment during 2015 and 2016.  As part of this analysis, this thesis introduces a new and more accurate technique for identifying whole ISR systems composed of multiple ISR jets.   The methods demonstrated in this analysis are completely generalizable and can be used in many other BSM searches and precision SM model measurements of the ISR $\pt$ spectrum.